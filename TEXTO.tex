\chapter*{}
\thispagestyle{empty}
\vfill{}
\begin{flushright}
\textit{Para Laura, que revoluciona minha vida e\\me ensina novas formas de resistência.}
\end{flushright}

\chapterspecial{Introdução}{A desinformação como\\estratégia de governo}{}
\markboth{Introdução}{}

\setlength{\epigraphwidth}{.60\textwidth}
\begin{epigraphs}
\qitem{\begin{flushright}A mentira se converte em ordem universal.\end{flushright}}{\begin{flushright}\textsc{franz kafka}\end{flushright}}
\end{epigraphs}

\noindent{}Este livro foi concluído em plena pandemia da \textsc{covid-19}, um dos momentos
mais trágicos da história, que transformou o modo de viver da
população pelo mundo. Também transformou a proposta inicial dessa
publicação. Fomos atravessados por esse acontecimento, e foi impossível
não colocá-lo em destaque.

O Brasil foi reconhecido internacionalmente como o país com a pior
gestão da pandemia, e o presidente considerado um dos mandatários
mais negacionistas e insensíveis frente ao imenso número de mortes
registrados no país.

A Internet e as redes sociais digitais passaram a ser, para muitos, a janela para o
mundo. Através dela eram realizadas atividades
profissionais, encontros com amigos e familiares, compras e vendas
\textit{online}, aulas, consultas médicas e acesso à cultura. A vida
estava circunscrita às telas de computadores e celulares, embora
a maior parte dos brasileiros já não pudesse mais se manter em
isolamento social, ao colocarem suas vidas em risco para garantir
o sustento da família.

As plataformas digitais foram fundamentais para diminuir o impacto do
distanciamento social. Mas foi também por meio delas que o 
compartilhamento de informações falsas, mentiras e boatos sobre a
pandemia cresceu. O país ocupou o triste lugar daquele onde mais se compartilhou
informações falsas ou duvidosas sobre o coronavírus.

O negacionismo se manifestou de diversas formas, e foi uma constante
durante a pandemia. Alimentou as redes de desinformação e criou
grande confusão entre os brasileiros sobre a conduta a ser adotada
para se proteger de um vírus extremamente contagiante.

A pandemia se tornou então uma espécie de pano de fundo para os discursos de
ódio, teorias conspiratórias, notícias falsas que proliferavam nas redes
produzindo um ambiente de desconfiança, medos e incertezas.

O primeiro capítulo do livro tem como eixo a reflexão sobre a desinformação na pandemia.
Foi inicialmente planejado para analisar o anticientificismo, negacionismo e
anti-intelectualismo. Mas a opção foi, por fim, pela discussão dos conceitos a partir do  
contexto da indústria de produção e disseminação de
informações falsas ou duvidosas, no cenário trágico de crescimento do
número de contaminados e de mortos. Durante o processo, foi identificado que a
pandemia se transformou em um palco para a disputa de narrativas em
torno da \textsc{covid-19}: além do debate com argumentos científicos, também um campo de disputa político, ideológico, sanitário, econômico e social.

No segundo capítulo, o uso da desinformação
na dinâmica política, particularmente nos processos eleitorais, é discutido.
E recupera o debate sobre a desinformação e as \textit{fake news}: o
papel das tecnologias de informação e comunicação na produção e
disseminação de informações responsáveis pelo abalo à dinâmica
democrática.

É também abordada a eleição de 2018. A apresentação das denúncias de
ilegalidades ocorridas durante o processo eleitoral, as investigações em
andamento, além divulgação contínua de informações
falsificadas e fraudulentas durante o governo de Jair Bolsonaro, 
que configurou o processo de institucionalização da prática desinformativa
por parte de lideranças políticas governamentais.

O terceiro capítulo fala sobre os principais conceitos e teorias acerca dos
termos \textit{fake news}, desinformação e pós-verdade --- aspecto fundamental para
qualificarmos o debate, tendo em vista por exemplo que \textit{fake news}, além de ser um
termo impreciso, é incapaz de explicar a complexidade do fenômeno.

Falaremos também sobre iniciativas para enfrentamento à
desinformação, que vão desde projetos de lei até campanhas de boicote do
financiamento a sites que produzem e compartilham \textit{fake news}. Além de  
iniciativas da sociedade civil, que atuam no combate ao
crescimento da produção e disseminação de mentiras.

Considerada a gravidade do fenômeno da desinformação,
e sua consequente demanda por um amplo debate social, as formas de enfrentamento verificadas em projetos, legislações,
plataformas, aplicativos e cartilhas são de grande importância. Da observação dos esforços
empreendidos pelos mais diversos setores da sociedade civil, veremos
como, apesar do momento inegavelmente conturbado que vivemos, é também 
necessário o reconhecimento da movimentação democrática para enfrentar o 
processo.

% Desejo que a leitura suscite a ampliação da reflexão e dos
% empreendimentos democráticos para enfrentarmos juntos a desinformação.

\chapter[Pandemia, desinformação e estratégia de morte]{Pandemia, desinformação e\\estratégia de morte}
\markboth{Pandemia, desinformação e estratégia de morte}{}

Maria Júlia era auxiliar de enfermagem de uma \textit{instituição de longa
permanência} (\textsc{ilp}) para acolhida de idosos e por esse motivo estava no
grupo prioritário para a imunização contra o novo coronavírus. Essa
decisão das autoridades de saúde partia do princípio da vulnerabilidade
desses profissionais, constantemente expostos ao vírus e mais
suscetíveis ao contágio.

A cuidadora recusou-se a tomar a vacina e ainda assim continuava
trabalhando, até que vieram os primeiros sintomas da doença:
febre, dor de garganta e pelo corpo. Não se preocupou,
achou que era uma gripe como qualquer outra. Lembrava-se das palavras do
presidente Jair Bolsonaro que desde o início da pandemia afirmava que
era uma gripezinha.

Passaram-se alguns dias e a auxiliar de enfermagem começou a sentir
falta de ar e decidiu fazer o teste para saber se havia sido
contaminada. O resultado foi positivo. Continuou tranquila e, mais uma
vez, resolveu seguir as orientações do que chamava o ``kit do nosso
presidente''. Começou a usar o medicamento precoce, à base de cloroquina
e ivermectina, e continuou em casa esperando pela melhora. Vale
ressaltar que o uso desses medicamentos era contestado por autoridades
sanitárias internacionais através de vários estudos que demonstravam sua
ineficácia para o tratamento do novo coronavírus.

Como a dificuldade para respirar aumentava, ela resolveu procurar um
pronto-socorro da rede pública. Os exames detectaram grave
comprometimento dos pulmões, motivo que a levou a ser internada. A
gravidade da situação de Maria Júlia inspirava os cuidados da \textsc{uti}, mas naquele momento o hospital não tinha leitos
disponíveis. Foi obrigada a permanecer por um período em uma maca
aguardando a liberação de uma vaga. Seu estado de saúde piorou, a ponto
de precisar ser entubada. Ficou inconsciente por alguns dias, até
melhorar o suficiente para continuar o tratamento em leito normal e,
finalmente, ter alta e retornar para casa.

Após uma crise de choro, declarou se sentir arrependida por negligenciar
a doença, entendendo que era, de fato, muito perigosa. Também se sentia
culpada por achar que poderia ter contaminado até mesmo seus pais e
outros idosos, o que de fato ocorreu. Enquanto estava na \textsc{uti}, seus pais
foram internados, intubados, mas não resistiram à doença e faleceram.

Histórias como essa foram contadas por profissionais da linha de frente
do enfrentamento à pandemia de \textsc{covid-19} e, desde 2020, passaram a fazer
parte cotidiana da cobertura midiática. Por outro lado, também
aumentaram muito os boatos, rumores e a desinformação sobre a doença,
disseminados pelas redes digitais. Estava em curso a disputa de
narrativas em torno da pandemia. Ao mesmo tempo que se verificava o
grande esforço das autoridades sanitárias internacionais para adotar as
medidas adequadas para garantir a saúde da população, autoridades
públicas, valendo-se das estratégias negacionistas e de teorias da
conspiração, contribuíam para gerar um ambiente de desconfiança na
população. E como era de se esperar, essa dinâmica repercutia no
comportamento individual e coletivo.

Era possível acompanhar os esclarecimentos de médicos sanitaristas,
infectologistas, virologistas, biólogos que passaram a ocupar as telas
das emissoras de \textsc{tv}, lives da Internet, microfones das emissoras de
rádio, páginas de jornais para explicar no que consistia a pandemia da
\textsc{covid-19}, seus riscos de contágio, possibilidades de tratamento, efeitos
colaterais nos pacientes que sobreviveram à doença. Enfim, desenvolvendo
um serviço de utilidade pública da maior importância para esclarecer a
população. Esses técnicos também tentavam desmentir boatos, notícias
falsas e informações descontextualizadas, no esforço de fazer com que a
população orientasse suas condutas baseadas em critérios estabelecidos a
partir de evidências, elementos empíricos em debate na comunidade
científica em âmbito internacional, que desenvolviam estudos para
compreender o comportamento~do~novo~coronavírus.

O alto volume de informações que circulavam sobre a pandemia fez com que
alguns pesquisadores passassem a chamar o fenômeno de \textit{infodemia}, ou
seja, um excesso de explicações que circulavam nas mais diversas mídias.
Não se trata apenas de notícias falsas ou desinformação, mas o volume
dificultava a seleção das orientações oferecidas por autoridades
sanitárias com base em evidências daquelas que não tinham nenhum
embasamento científico. Era um cenário propício para a disseminação de
notícias falsas e desinformação, agravando o desafio de enfrentar os
impactos causados pela doença.

Essa preocupação também mobilizava as autoridades sanitárias, como a
gerente de mídias sociais da Organização Mundial de Saúde (\textsc{oms}),
Aleksandra Kuzmanovic, que acompanhava Facebook, Twitter, Pinterest e
Google para garantir que usuários dessas redes tivessem acesso às
informações oficiais.

\begin{quote}
Conforme declarado pela \textsc{oms}, o surto de \textsc{covid-19} e a resposta a ele têm
sido acompanhados por uma enorme infodemia: um excesso de informações,
algumas precisas e outras não, que tornam difícil encontrar fontes
idôneas e orientações confiáveis quando se precisa {[}\ldots{]} Nessa situação,
surgem rumores e desinformação, além da manipulação de informações com
intenção duvidosa. Na era da informação, esse fenômeno é amplificado
pelas redes sociais e se alastra mais rapidamente, como um vírus.\footnote{Organização Pan-Americana da Saúde. \textit{Entenda a infodemina e a desinformação na luta contra o \textsc{covid-19}}. Relatório do Departamento de Evidência para ação em saúde, 2020.}
\end{quote}

A infodemia no Brasil se materializava na enxurrada de desinformação,
tendo como importante polo emissor o governo federal, que passava a maior 
parte do tempo se dedicando à produção e disseminação de mentiras e informações
destorcidas sobre a pandemia. Tempo que se tivesse sido dedicado ao
esclarecimento dos riscos trazidos pelo vírus e na busca de soluções
adequadas para enfrentá-lo, certamente, teríamos um cenário com menos
mortes.

\section{Negacionismo científico e a realidade paralela}

Segundo ranking desenvolvido pela Universidade de HopKins, dos Estados
Unidos, o Brasil detinha o recorde mundial no compartilhamento de peças
de desinformação sobre o número de mortos e infectados pela \textsc{covid-19}.
Esse levantamento foi realizado com base nas plataformas de checagem de
informações para contestar o discurso de negação sobre a doença.

O negacionismo científico gera diversionismo, cria ambiente de perda de
credibilidade nos procedimentos científicos, proporcionando uma espécie
de suspensão dos parâmetros de realidade. E, frequentemente, gerando uma
ambientação característica de realidade paralela, uma espécie de aversão
ao conhecimento e apologia à ignorância, desconsidera fatos --- ou, como
dizia o ex-presidente Trump, apresentar fatos alternativos.

As diversas teorias da conspiração que alimentam os processos de
desinformação estão cada vez mais vinculadas aos grupos
ultraconservadores e extremistas de direita, motivados pela cruzada
anticiência e na deslegitimação do papel da pesquisa científica. No caso
das campanhas antivacina é possível verificar o estímulo à desconfiança
na comunidade médica, por meio de mensagens e campanhas difamatórias.

Evidentemente, a ciência tem falhas, apresenta controvérsias e pode
cometer erros em estudos e condutas adotadas. Portanto, pode e deve ser
problematizada em seus métodos e práticas, até mesmo porque é esse
processo que amplia o conhecimento científico. A cloroquina é um exemplo
desse processo de investigação científica. Em uma primeira fase de
testes, chegou a ser cogitada por alguns profissionais da área da saúde
como medicamento indicado para o cuidado de pessoas infectadas pela
\textsc{covid-19}. Com a continuidade das pesquisas e após debates, estudos,
evidências empíricas, verificou-se que não somente ela não era eficaz
para esse tipo de tratamento como também pode ser perigosa, trazendo
efeitos colaterais, por exemplo, em pacientes com problemas cardíacos.

Podemos entender o negacionismo científico como a forma de se recusar
evidências da realidade empírica. Trata-se da rejeição de conceitos
básicos, construídos por meio dos debates estabelecidos pela comunidade
científica em detrimento de ideias controversas, radicais e de teorias
da conspiração.

Na mesma perspectiva, o anti-intelectualismo expressa um sentimento
oponente em relação a pesquisadores e cientistas. Uma ação comum é o
ataque à ciência, à educação e à arte, com argumentos extremistas que
exploram um sentimento de hostilidade em relação ao conhecimento
científico, criando um falso ambiente de rivalidade.

As motivações vão desde crítica a intelectuais, aos métodos da pesquisa
científica até acusações mais injustificadas sobre a existência de
linhas de financiamento à investigação, principalmente no âmbito
acadêmico. Uma situação que ilustra perfeitamente o negacionismo
científico por parte do governo federal e parlamentares do Congresso
Nacional que fazem parte da base de apoio do presidente foi o debate
sobre a destinação de recursos orçamentários para a realização do Censo
Demográfico. Realizado de dez em dez anos, trata-se de um levantamento
fundamental para o conhecimento da realidade socioeconômica de um país e
serve de base para a elaboração de políticas públicas, entre diversas
outras ações.

O orçamento de 2021 não destinou verba para a realização desse
levantamento e o Brasil corre o risco de viver uma espécie de apagão
estatístico. Ao mesmo tempo que o corte era realizado, Bolsonaro
praticava o bom e velho ``toma lá, dá cá'', liberando bilhões em verbas
para que parlamentares utilizassem em suas regiões, mantendo os
eleitores em troca de obras cuja finalidade poderia ser questionada em
muitos casos.

É importante ressaltar que além do conhecimento científico e seus
parâmetros e regras próprias, também consideramos legítimos outros
saberes como aqueles praticados por povos tradicionais, tais como
indígenas e quilombolas. Os saberes desses povos são muito importantes
para seus modos de vida e é fundamental pensarmos que o conhecimento
científico não deve se sobrepor ao conhecimento popular. Devem,
portanto, ser saberes aliados com o objetivo de garantir a melhoria das
condições de vida das populações.

\section{A produção política e cultural da ignorância}

Robert Proctor, professor de História da Ciência da Universidade de
Stanford, nos \textsc{eua}, afirma que há uma forma de produção política e
cultural da ignorância, a agnotologia, conforme definiu. Alguns grupos
econômicos e políticos se beneficiam da ignorância social e as
populações se tornam facilmente moduláveis por informações duvidosas.
Significa dizer que o culto à ignorância não está necessariamente
relacionado à falta de escolaridade, mas principalmente a interesses
sociais, econômicos e políticos que pretendem manter parcela da
população cada vez mais conectada a realidades paralelas.

As práticas anticientificistas não são exatamente novidades, ainda menos
no campo da saúde, onde é frequente o fenômeno do compartilhamento de
informações ou ``dicas'' milagrosas sobre cuidados para se evitar ou
cuidar de determinadas doenças. Tudo isso, presente há algum tempo no
debate social, ficou ainda mais acentuado com o surgimento da \textsc{covid-19}.

De certa forma, a humanidade foi surpreendida pela pandemia. Um
acontecimento distópico, típico dos filmes hollywoodianos de catástrofe,
que tantas vezes representaram nas telas pandemias que praticamente
extinguiriam a humanidade. Há tempos alguns cientistas anunciavam a
possibilidade de vivenciarmos um impacto nos modos de adoecimento, nas
mutações de vírus da gripe e nas mudanças climáticas, alertando-nos
sobre a possibilidade de vivenciarmos um processo de contágio em escala
planetária que viria a incidir em nossos modos de vida, nas formas de
produzir e de consumir.

A pandemia acionou um cenário quase épico, cuja memória da população
global se limitava às narrativas encontradas em livros de história, na
literatura, no cinema, nas artes plásticas e em algumas poucas pessoas
centenárias.

Uma pandemia ocorre quando uma epidemia foge ao controle local e passa a
afetar grandes contingentes populacionais, podendo atingir o planeta
como um todo, como no caso da \textsc{covid-19}. Historicamente, as gripes já se
tornaram pandêmicas. No início do século \textsc{xx}, a gripe espanhola
contaminou 50\% da população mundial e matou aproximadamente 5\% em um
período de aproximadamente dois anos. Tudo indicava que o vírus veio da
Europa em navios e desembarcou no país com pessoas já contaminadas.

As pandemias têm origem com bactérias e vírus. Especificamente
no caso dos vírus, eles podem circular facilmente entre espécies
diferentes, possuem maior capacidade de adaptação e por isso podem se
tornar pandêmicos. Por exemplo, o vírus que causou a gripe
aviária teve origem nas aves, a gripe suína veio dos porcos e,
provavelmente, o ebola teve origem no morcego. Além disso, há os
mosquitos transmissores. No caso do Brasil, a dengue, zika e
chicungunha, são transmitidos pelo \textit{Aedes aegypti}, o mosquito-da-dengue.

O surgimento de uma pandemia já era previsto pela comunidade científica
devido a uma série de mutações de alguns vírus e pelas mudanças
ambientais e climáticas. Em maio de 2019, a plataforma Netlix lançou uma
série chamada ``Pandemia'' e é surpreendente como um conjunto de
cientistas já alertava a proximidade de um novo tipo de gripe que
poderia contaminar a humanidade. Na série também apareceu o desafio
imposto pelas teorias da conspiração através das notícias falsas como no
surto de ebola nos países africanos. Vários boatos disseminados
afirmavam que a vacina era parte de um plano dos Estados Unidos para
matar a população, fazendo com que muitos deixassem de se imunizar e
contraíssem a doença.

Os vírus, no geral, possuem um comportamento complexo e necessitam de
diversos condicionantes para se instalarem e proliferarem. Não é
objetivo abordar em profundidade essa temática, mas destacar alguns
aspectos que fazem com que a pandemia sinalize elementos fundamentais
para pensarmos sobre os modos de vida da contemporaneidade, tais como a
forma como vivemos, produzimos, consumimos, nos deslocamos e descartamos
resíduos sólidos, parte de uma conduta que vem alterando de forma
destruidora os recursos naturais.

O médico Stefan Cunha Ujvari, em sua obra \textit{História da
humanidade contada pelos vírus}, demonstra a centralidade do papel dos
microrganismos na História, apresentando a inter-relação entre a
genética e as ciências humanas e sociais. Trata-se de uma reflexão
fundamental para se compreender essa convivência, seus impactos e
riscos, tendo em vista que esse encontro provoca o que o autor denomina
por ``estilhaços microscópicos deixados no organismo invadido'',
transformando a dinâmica dos corpos e, inevitavelmente, as próprias
sociedades.

Entre as medidas adotadas por governantes em diversos países estão o
mapeamento e a modelagem da propagação do vírus a partir do
monitoramento dos comportamentos por meio dos dispositivos tecnológicos.
Isso propiciou que em alguns lugares fosse possível a realização do
confinamento de pessoas que apenas haviam estado nos mesmos lugares que
pessoas contaminadas, como por exemplo o transporte público. Aqui fica a
tensão, o paradoxo entre a adoção de medidas sanitárias para conter a
escalada do vírus e a manutenção da privacidade afetada pelas
tecnologias de controle individual e coletivo e, normalmente, a
preocupação com o bem-estar coletivo impõe decisões que futuramente
podem ser utilizadas para outros fins.

Alguns países atuaram mais rápido e de forma mais eficaz justamente
porque conseguiram conter o número de contaminados e de mortes. Outros,
como o Brasil, não vivenciaram a mesma situação. Infelizmente, o país é
considerado com um dos piores no gerenciamento da pandemia. Isso ocorre
não somente pelo conjunto de deficiências no sistema de saúde, embora os
princípios norteadores de seu funcionamento sejam bastante avançados,
mas principalmente pela desastrosa gestão do governo federal.

\section{Desinformação circula como numa câmara de eco}

O governo de Jair Bolsonaro investiu sistematicamente na desinformação,
distorcendo a realidade, gerando confusão na população em relação às
medidas para a prevenção do contágio e, principalmente, na recomendação
do tratamento precoce. Mesmo com as diretrizes da Organização Mundial da
Saúde (\textsc{oms}), que não recomendava a adoção desses medicamentos para o
tratamento da \textsc{covid-19}, Bolsonaro continuava estimulando a população a
fazer uso do tratamento precoce, além de desobedecer a outras medidas de
prevenção, como por exemplo o não uso de máscaras, provocar aglomeração
e incentivar a desobediência às medidas de isolamento social
determinadas por prefeitos e governadores.

Em lives, programas realizados semanalmente por um canal do YouTube,
Bolsonaro mais parecia um propagandista da cloroquina. Nunca desistiu de
defender o tratamento precoce, que consistia no uso de cloroquina e
ivermectina (as mais populares), nitazoxanida, zinco,
remdesivir, azitromicina. Chegou até mesmo a comprar insumos para a
fabricação do medicamento pelo laboratório do exército. Bolsonaro sempre
afirmava que havia consultado médicos de diversas especialidades que
também recomendavam essa prescrição. Nesse caso específico, é importante
destacar que o negacionismo também opera com profissionais de dentro do
campo científico, atuando para dar legitimidade à narrativa
presidencial.

A \textsc{oms} contestava esse tipo de desinformação e se posicionava
contrariamente à adoção do medicamento para o combate ao vírus. Além de
reafirmar não haver estudos científicos que comprovassem sua eficácia,
também alertava para os possíveis efeitos colaterais, principalmente em
pessoas que já apresentem algum diagnóstico de doenças cardíacas.\footnote{Felipe Grandin. É \textsc{\#fake} que pesquisa recente indique a
hidroxicloroquina como o tratamento mais eficaz contra o coronavírus. \textit{G1}, 21 de maio de 2020.}

Como prática da estratégia bolsonarista de disseminar a desinformação
quando alguma polêmica era difundida pelo presidente, imediatamente sua
rede de apoiadores era acionada para ampliar o impacto de sua narrativa.
E, em boa parte dessas situações, o instrumento utilizado por esses
grupos são as notícias falsas, distorcidas e sem base em fatos. Além da
mensagem circular nas redes sociais, a mesma informação foi compartilhada
pela Secretaria Especial de Comunicação Social (\textsc{secom}) da Presidência da
República, afirmando que a cloroquina era o tratamento com maior
eficácia para o combate da \textsc{covid-19}. Trata-se do uso da estrutura de
comunicação presidencial para propagar a desinformação, ao mesmo tempo que
nenhum órgão governamental realizava campanhas de orientação e
esclarecimentos sobre a pandemia e suas formas de contágio.

A tática de compartilhamento da desinformação nas redes bolsonaristas
pode ser associada à noção de câmaras de eco. Ao compartilhar dentro de
um grupo de apoiadores do presidente informações positivas, por exemplo,
sobre o uso da cloroquina, elas são amplificadas dentro de um grupo, o
que ocorre com frequência no ambiente digital, fazendo com que os
usuários tenham acesso majoritariamente àquilo que circula dentro da sua
própria bolha.

\section{Pandemia é mais letal nos bolsões bolsonaristas}

Um estudo realizado por economistas da Fundação Getúlio Vargas (\textsc{fgv}) em
parceria com a Universidade de Cambridge, no Reino Unido, e divulgado
pela agência de checagem Aos Fatos, mostra que as declarações públicas
do presidente Bolsonaro levaram à queda de isolamento social entre seus
apoiadores. O estudo foi realizado com base na divisão 
de municípios brasileiros pró e anti-Bolsonaro, a partir dos resultados
das eleições de 2018. Em seguida foram utilizados dados de localização
de celulares para comparar o nível de adesão às medidas restritivas de
circulação da população em espaços públicos.%ñ está na bibliografia: (FAVERO B, 2020)

As cidades com maior presença de eleitores bolsonaristas apresentaram as
menores taxas de isolamento. Na opinião dos pesquisadores, os dados são
capazes de mostrar que o discurso de lideranças políticas produz grande
impacto na forma como as pessoas reagem à pandemia. O poder discursivo
de um presidente exerce força frente à população, porém não é somente o
que ele fala, mas sua conduta cotidiana, seus movimentos, seus gestos.
São em momentos tão dramáticos quanto os de uma pandemia que podemos
reconhecer as qualidades de uma líder na condução dos desafios de
caráter social, político e sanitário.

Pesquisa semelhante realizada pela Universidade Federal do Rio de
Janeiro (\textsc{ufrj}) identificou o chamado \textit{efeito Bolsonaro} na
disseminação do vírus pelas cidades do Brasil. A pesquisa divulgada em
outubro de 2020 realizou um cruzamento de dados com o número de casos de
contaminação do vírus e o resultado do primeiro turno das eleições de 2018
nos 5.570 municípios brasileiros. Os pesquisadores destacaram que a cada
10 pontos percentuais de votos para Bolsonaro há um aumento de 11\% de
infectados e 12\% no número de mortos.

O chamado efeito Bolsonaro pode ser interpretado como influência em seus
apoiadores que ao seguirem a conduta de desrespeito às medidas de
distanciamento social e ao uso de máscara, adotavam um comportamento de
risco e por isso se tornavam mais vulneráveis ao contágio.

Com um discurso voltado para a recuperação econômica, em uma de suas
declarações públicas, registrada no site \textsc{uol}, Bolsonaro disse que a
população deveria enfrentar ``o vírus como homens e não como moleques''.\footnote{Adriana Ferraz. \textit{Uol}, 29 de março de 2020.}
A narrativa presidencial tinha como objetivo não
permitir que as pessoas identificassem suas dificuldades econômicas com
a ausência de uma política pública de assistência social. Desse modo,
Bolsonaro se isentava da responsabilidade do setor público e despejava
nos indivíduos a responsabilidade de produzir as condições para a vida
de suas famílias.

\section{Pandemia expõe limites das políticas neoliberais}

Antes mesmo do início da pandemia, a situação econômica do país já era
grave e piorava expressivamente. No segundo semestre de 2020 o país
tinha aproximadamente 40\% de trabalhadores na informalidade, ou seja,
sem vínculo formal e sem benefícios como férias, 13º salário, fundo de
garantia e aposentadoria, segundo dados do \textsc{ibge}. Esse contingente teve que diariamente
sair às ruas desempenhando as mais diversas atividades, colocando a vida
em risco para garantir minimante o sustento de suas famílias.

A individualização de responsabilidades econômicas é parte fundamental
da estratégia discursiva capitalista e ainda mais na era neoliberal, na
qual os indivíduos são influenciados a construir uma autoimagem de
empreendedores de si mesmos, portanto responsáveis pela satisfação de
suas necessidades.

A pandemia expôs claramente o limite das políticas neoliberais e
governos de diversos países se encontraram na encruzilhada da história
tendo que rapidamente adotar medidas de cunho no mínimo socialdemocrata,
reativando em alguma medida aspectos da era do Estado de bem-estar
social e atuando para suprir necessidades básicas para a garantia da
manutenção e reprodução da vida. Ou seja, o Estado tinha que se fazer
presente, mesmo naquelas localidades em que o discurso neoliberal é
hegemônico desde os anos 1990.

O Brasil caminhava na contramão dessa realidade e o presidente não
perdia oportunidade para atacar prefeitos e governadores que adotavam
medidas restritivas de deslocamento social, disseminando desinformação e
discurso negacionista em relação às orientações de instituições
nacionais e internacionais vinculadas à saúde.

O crescimento das taxas de desemprego em países como o Brasil aumentava
ainda mais a condição precária de parcela expressiva da população. A
retração da economia imposta pela pandemia teve impacto maior na classe
trabalhadora e nos informais. Isso refuta a ideia que circula no campo
social de que o vírus é democrático, afeta a todos igualmente. O
contágio é maior em regiões com moradias precárias, nas quais as medidas
de prevenção não são facilmente adotadas. Regiões em que há falta de
água e saneamento básico, onde famílias inteiras compartilham apenas um
cômodo entre muitos moradores. Ou seja, a pandemia expunha ainda mais as
desigualdades sociais do país: nessas regiões o contágio é maior que em
outras partes dos centros urbanos.

O transporte público é o espaço com maior possibilidade de
contaminação e, mais uma vez, a população pobre e periférica é mais
vulnerável, considerando que, no geral, concentram-se nas regiões
distantes dos centros urbanos, o que lhes obriga a passar horas
no transporte público que apresenta lotação sempre acima da demanda. As
aglomerações começam a ser formadas nas estações ou terminais e as cenas
de concentração de pessoas à espera do transporte são frequentes. Não há
alternativa para uma multidão que precisa se deslocar e acaba arriscando
a vida cotidianamente.

O presidente Jair Bolsonaro insistia em minimizar a gravidade da
situação e, frequentemente, afirmava que o desemprego era pior que o
vírus, ou seja, demonstrava maior preocupação com a economia do que com
a vida da população. Com essa argumentação reiterava sua campanha contra
as medidas de isolamento social, defendendo a abertura total do
comércio, das escolas, a volta ao trabalho e, supostamente, à
normalidade. Simultaneamente, vídeos dos apoiadores do presidente
viralizavam nas redes numa campanha de mobilização nacional pela
reabertura de todos os serviços, cenas mostrando o presidente saindo às
ruas, entrando em comércios e até mesmo comendo o típico espetinho
vendido por trabalhadores informais. Além de querer se mostrar como um
homem do povo, que se arriscava, que não era covarde, gerava uma falsa
imagem da pandemia, diferente do cenário dos hospitais superlotados
mostrado~pelos~meios~de~comunicação.

Após meses de discussões foi aprovado o auxílio emergencial.
Inicialmente, o ministro da Economia propunha o valor de 200 reais,
afirmando ser suficiente para garantir os custos da cesta básica,
demonstrando o desconhecimento absoluto da realidade do país. A proposta
foi fortemente criticada pelos setores da oposição ao governo e após
mobilização pelas redes digitais foi possível reverter os valores
iniciais e se chegou à concessão do valor de cinco parcelas de 
600 reais e 1.200 reais para mães chefes de família.

No entanto, parte expressiva dos trabalhadores que se enquadram no
perfil dos beneficiários demorou muito para receber o benefício ou não
conseguiu acessá-lo, demonstrando incompetência para a implantação e
alto grau de descaso por parte das autoridades governamentais. A
população se via desorientada com a burocratização para o recebimento, a
falta de informações claras e a sobrecarga do atendimento dos
funcionários da Caixa Econômica Federal.

Um sistema ineficaz e gerador de aglomerações nas portas das agências
bancárias, contrariando as diretrizes de isolamento social para evitar
aglomerações.

\section{Distopia que remete à necropolítica}

Esse era um cenário distópico que remetia à noção de \textit{necropolítica},
descrita pelo historiador camaronense Achille Mbembe.\footnote{\textit{Necropolítica}. São Paulo: n-1 edições, 2018.} Inspirado na
noção de biopolítica de Michel Foucault, o autor passou a pensar na
necropolítica, o poder de definir quem pode viver e quem deve morrer.

Com base em dispositivos e tecnologias de poder que fazem a gestão e o
controle das populações, o simples ``deixar morrer'' passa a ser
aceitável, deixando de causar estranhamento. A reflexão elaborada à luz
do estado de exceção, do terror, demonstra como o Estado adota uma
política de morte e que o uso legítimo da força, a relação
amigo-inimigo, são a base que caracterizava o extermínio de parcela da
população, invariavelmente os mais vulneráveis, a saber, pobres,
periféricos e negros, os chamados corpos matáveis, que estão em risco a
todo instante. Essa cena também era observada em hospitais, quando
médicos tinham que decidir para qual paciente deveriam colocar o
oxigênio, situações vividas em várias partes do país e, principalmente,
em Manaus.

Em entrevista ao jornal \textit{Folha de S.\,Paulo}, Mbembe\footnote{Pandemia democratizou poder de matar, diz autor
da teoria da \textit{Necropolítica}. \textit{Folha de S.\,Paulo}, 30 de março de 2020.} aborda a
necropolítica no neoliberalismo e vemos claramente essa política de
morte na atuação do governo brasileiro durante a pandemia com a lógica
de priorizar a economia em detrimento da saúde da população. A campanha
do presidente contra o isolamento social articula-se perfeitamente à
análise do historiador:

\begin{quote}
Essa é a lógica do sacrifício que sempre esteve no coração do
neoliberalismo, que deveríamos chamar de necroliberalismo. Esse sistema
sempre operou com um aparato de cálculo. A ideia de que alguém vale mais
do que os outros. Quem não tem valor pode ser descartado. A questão é o
que fazer com aqueles que decidimos não ter valor. Essa pergunta, é
claro, sempre afeta as mesmas raças, as mesmas classes sociais e os
mesmos gêneros.\footnote{\textit{Idem.}}
\end{quote}

Com dificuldades para satisfação das necessidades básicas e sem receber
o auxílio emergencial, milhões de brasileiros arriscavam suas vidas
descumprindo as regras do isolamento social decretado por prefeitos e
governadores. Indo ao sacrifício, conforme Mbembe. Sacrifício que coloca
em risco a própria vida, a da família e da comunidade em geral.

O vaivém do auxílio emergencial, que em 2020 durou meses, e a
descontinuidade do benefício foi responsável pelo aumento da fome no
país. Segundo dados da Fundação Getúlio Vargas, a pobreza triplicou
entre agosto de 2020 e fevereiro de 2021, aumentando de 9,5 milhões para
27 milhões de pessoas. Mas essa realidade não despertava a sensibilidade
do governo federal, que se mostrava mais preocupado com as planilhas do
que com as pessoas em situação de miserabilidade.

A taxa de desocupação também cresceu entre 2020 e início de 2021. Chegou
a 14,4\% da população segundo o \textsc{ibge}, a maior taxa desde que a Pesquisa
Nacional por Amostra de Domicílios Contínua (\textsc{pnad}) começou a ser
realizada em 2012. Isso significava que a economia não dava sinais
expressivos de recuperação e cada vez ficava mais claro que a situação
somente começaria a mudar caso a imunização da população fosse realizada
para que a volta a algum nível de normalidade pudesse ocorrer. Era a
única forma para enfrentar o aumento da desigualdade, combinar
investimentos na economia com a vacinação em massa.

Desde o início da pandemia Bolsonaro não perdia a oportunidade de
construir sua narrativa de responsabilização de prefeitos e governadores
em relação aos efeitos econômicos trazidos pela pandemia e as medidas de
distanciamento e isolamento social adotadas em algumas regiões.
Desconsiderava qualquer argumentação sanitária que defendia essas
medidas como forma para conter o avanço dos casos e, além da falta de
leitos hospitalares para tratar as pessoas que manifestassem sintomas
graves da doença. Assim, o aumento crescente do número de pessoas
contaminadas, o crescimento da média diária de mortos, hospitais
lotados, falta de leitos de \textsc{uti} e o colapso funerário formaram um
cenário trágico para a população brasileira, sobretudo se considerarmos
que parte dessas mortes poderia ser evitada se não fosse a narrativa
desinformativa e uma política genocida adotada pelo governo federal.

A estratégia presidencial se baseava na negação do número de mortes
geradas pela pandemia. Não faltavam peças desinformativas nas redes
sociais sugerindo que havia interesses por parte de prefeitos e
governadores em inflar o número de mortos causados pelo novo
coronavírus, tendo em vista que quanto maior o número de mortos em uma
região, mais aumentavam as verbas a serem recebidas do governo federal.
Dizia que havia até alteração do registro de mortes por outras causas,
contabilizadas como mortos pela \textsc{covid-19}, segundo ele favorecendo assim
governantes inescrupulosos que só tinham interesse em receber mais
recursos federais.

O presidente chegou a mencionar um documento do Tribunal de Contas da
União que contestava as notificações de mortes por \textsc{covid-19}, confirmando
a narrativa bolsonarista de que a pandemia estava superdimensionada. No
mesmo dia, o \textsc{tcu} abriu uma investigação para apurar a informação e
identificou que o documento havia sido produzido por um técnico que
teria forjado os dados. O funcionário foi afastado para garantir que não
interferiria nas investigações, mas já se sabe que faz parte da rede de
bolsonaristas. O fato demonstrou que não se trata somente da
disseminação de informações falsas por parte de um núcleo único
centralizado, embora existam muitas evidências sobre uma ação
orquestrada por um comando. Verifica-se uma espécie de capilarização da
falsificação de informações oficiais com o objetivo de dar sustentação
ao governo.

Bolsonaro em inúmeras aparições públicas fez declarações que
corroboravam esse tipo de boato, demonstrando que também não concordava
com os números e as informações veiculadas pela mídia. Algumas vezes
afirmava que a mídia era comunista e tinha como objetivo atingir a
imagem de seu governo, que fazia parte das forças políticas que tinham
interesse em retomar o poder para continuar com a ``roubalheira'' de
sempre. Reiteração da criação de inimigo comum a ser combatido.

É importante esclarecer que uma portaria de julho de 2020 estabelecia os
critérios de repasses dos recursos federais para as ações municipais e
estaduais de enfrentamento à pandemia, mas não há nenhuma menção que
essa transferência de recursos deveria ser vinculada ao número de mortos
pelo novo coronavírus.

No Brasil, os gastos com saúde diminuíram expressivamente quando a área
mais precisava de aumento de recursos para o atendimento das demandas da
população. Em 2020, o orçamento executado foi de 160,9 bilhões,
incluindo os créditos da pandemia. Para 2021 foram aprovados 125,8
bilhões, ou seja, um corte de aproximadamente 35 bilhões, reduzindo
a capacidade de atendimento do \textsc{sus} e, consequentemente, trazendo mais
impactos para o sistema.

A partidarização da pandemia estava dada e, em diversas oportunidades, o
conflito entre o presidente e alguns governadores aumentava. João Doria
(\textsc{psdb}), governador do estado de São Paulo, era um dos mais atingidos.
Bolsonaro acusava o governador de autoritário por querer impor as
medidas de isolamento social, impedindo as pessoas de trabalharem. Um
dos \textit{memes} mais compartilhados pelas redes bolsonaristas trazia a imagem
de Dória associada a Adolph Hitler, para criar a associação a um dos
maiores genocidas da história.

\section{Desinformação como estratégia de governo}

A postura negacionista do presidente e de seus apoiadores mais fiéis
dentro e fora do governo faz parte da intencionalidade de confundir a
população sobre as medidas sanitárias e gerar um ambiente de
desconfiança generalizada na ciência e nas medidas protetivas adotadas
por prefeitos e governadores. Essa postura demonstra claramente que
desde janeiro de 2019, início do governo Bolsonaro, a desinformação se
transformou em estratégia governamental, orquestrada pelo presidente e
por parte de seus ministros, assessores e de seus filhos --- o senador
Flávio Bolsonaro, o deputado federal Eduardo Bolsonaro e o vereador pelo
município do Rio de Janeiro Carlos Bolsonaro.

Durante a pandemia, o presidente dedicou seu tempo para criar confronto
com os ministros da Saúde por ele próprio nomeados, mas que demonstraram
discordância, por exemplo, em relação ao tratamento
precoce. A situação desses ministros chegou ao ponto de serem
publicamente confrontados pelo presidente. Dois deles, Henrique Mandetta
e Nelson Teich, sistematicamente desautorizados, deixaram o governo em
plena pandemia, momentos em que se verificava o aumento do número de
mortes e de contaminados, o que requeria a necessidade de um comando
central para a rápida adoção de medidas para enfrentar o vírus. Ambos os
ministros mostravam um certo constrangimento com a insistência do
presidente em relação ao medicamento, mesmo que de forma tímida, sem
coragem para enfrentá-lo publicamente.

No lugar desses ministros foi colocado um general, Eduardo Pazuello, que
permaneceu aproximadamente dez meses no cargo, destacando-se por
obedecer fielmente ao presidente e contrariar todas as medidas aprovadas
pela comunidade científica nacional e internacional e agindo
desastrosamente nas situações mais críticas da pandemia, como a falta de
insumos necessários para as atividades hospitalares.

Pazuello era um ministro de fachada, um fantoche do presidente, não
atrapalhava seus planos e se subordinava à liderança do chefe, sem
nenhum tipo de contestação. A situação desconstruía a abordagem de
vários estudos sobre os militares e o papel das Forças Armadas no
governo Bolsonaro. Entre elas, descartamos que era o plano dos militares
fazerem a contenção aos arroubos autoritários do presidente. Se isso não
havia ocorrido durante a pandemia, pela gravidade da situação, ficava
difícil imaginar que poderia ocorrer em outro momento.

\section{Brasil de Bolsonaro contra a vacinação}

Enquanto o mundo aguardava ansiosamente pelo desenvolvimento de uma
vacina capaz de deter a expansão da pandemia, o debate tomava outros
rumos no Brasil. Um mapeamento realizado pela \textsc{oms} mostrava que
aproximadamente 160 experimentos estavam sendo realizados em 2020 por
diversos laboratórios para se chegar à descoberta de um imunizante eficaz. 
A intensificação das pesquisas mostrava o esforço da
comunidade científica, apoiada por muitos governantes de diversos
países, menos aqueles comandados pela extrema-direita, que insistia em
criar uma narrativa para gerar desconfiança na população.

O presidente Bolsonaro passou o ano de 2020 se posicionando
contrariamente à vacinação, defendendo a chamada imunização de rebanho.
Para que isso ocorresse, aproximadamente 70\% da população precisaria
ser infectada. Essa tese defendida pelo presidente era refutada por
infectologistas e virologistas, defensores da imunidade através da
vacinação em massa.

A tese da imunização de rebanho traria um problema a mais na complexa
realidade da pandemia em pelo menos dois aspectos centrais. O primeiro,
estava relacionado à ampliação do número de infectados, que gera o
aumento do número de mortos, colapso do sistema de saúde,
impossibilitando o atendimento aos que tivessem complicações da doença e
que precisassem de atendimento hospitalar. Outro fator preocupante era o
descontrole do vírus, favorecendo o surgimento de mutações e novas cepas
com comportamentos imprevisíveis. No caso brasileiro, isso ocorreu. Uma
variante foi detectada em Manaus e rapidamente se espalhou pelo país e
fora dele. O Brasil vivia as barreiras sanitárias impostas por vários
países, que impediam a entrada de pessoas procedentes do território
nacional, para conter a propagação do vírus.

Em meados de 2020, o anúncio do registro da primeira vacina era motivo de
comemoração e esperança, tendo em vista a dificuldade de se produzir um
imunizante eficaz em tempo tão curto de pesquisa. Era a notícia mais
esperada pela imensa maioria da população global, menos pelos ativistas
dos movimentos antivacina, interessados em ativar as mais diferentes
teorias conspiratórias, produzindo e disseminando pelas redes digitais
diversos \textit{memes} e textos semeando dúvidas sobre a eficácia do imunizante.
Tudo isso era estimulado pelo discurso do presidente.

Entre os alvos das campanhas de difamação estava o governador paulista
João Doria (\textsc{psdb}). As negociações com um laboratório chinês para o
desenvolvimento da vacina colocaram o governador em destaque no debate,
transformando-o em alvo da milícia digital bolsonarista, que rapidamente
criou um \textit{meme} acusando o governo de querer importar a vacina chinesa que
teria um chip acoplado capaz de controlar o comportamento da população.
A viralização foi imediata e desencadeou inúmeros boatos, mesmo antes da
existência do imunizante, mas preparando o caminho para a criação da
resistência da população em relação a ele.

Durante as eleições de 2018, conforme crescia a possibilidade de eleição
de Bolsonaro, candidato ao governo do estado de São Paulo pelo \textsc{psdb}, João
Dória abandonou a candidatura presencial de seu partido (Geraldo
Alckmin) e associou-se à campanha de Bolsonaro. Criou a chapa
Bolsodoria, aproveitando a onda conservadora para pegar carona no
processo eleitoral, comportamento tipicamente oportunista.

O Centro de Contingência do estado de São Paulo, criado para monitorar e
coordenar as ações contra o novo coronavírus, ocupou um papel importante
na adoção de medidas de prevenção e na negociação com a China para a
produção do imunizante no Instituto Butantan, reconhecido
internacionalmente por sua competência nesse ramo.

Outro alvo importante dos negacionistas era, como sempre, a China.
Bolsonaro também não desperdiçava nenhum instante para difundir sua
narrativa ideológica, carregada de discurso de ódio, e os chineses eram
o centro preferido de ataque. O fato de a epidemia ter eclodido na China
ativou o preconceito xenofóbico em relação ao país, além da crítica
constante ao regime comunista. Não é menos importante pensar que esse
preconceito era ainda mais preocupante pelo fato de estar hostilizando
uma das mais importantes potências do mundo e o maior parceiro comercial
do Brasil.

A diplomacia brasileira na era Bolsonaro passou a ter relações estreitas
não com os \textsc{eua}, mas com o ex-presidente Donald Trump, que não perdia
nenhuma oportunidade de atacar a China. Não faltaram notícias
fraudulentas e conspiratórias afirmando que o vírus havia sido criado em
um laboratório chinês, notícia desmentida pela comunidade científica
internacional que demonstrava, através de estudos, que não é possível
fabricar um vírus com as características da \textsc{covid-19}.

Levantamento realizado pela agência de checagem Pública mostrava que a
hashtag \#VirusChines alcançou os trending topics no Twitter em 19 de
março de 2020, no início da pandemia. Com o auxílio de robôs e também de
influenciadores digitais coordenados pelo deputado federal Eduardo
Bolsonaro (\textsc{psl}), filho do presidente, as redes digitais foram
inundadas por essa mentira sobre a origem do vírus.

Outra mentira que circulou em grupos de WhatsApp era sobre a declaração
do imunologista japonês Tasuku Honjo, vencedor do Prêmio Nobel de
Medicina em 2018, que também teria declarado que o novo coronavírus era
artificial e havia sido criado pela China. Rapidamente, a agência de
checagem de informação Aos Fatos verificou que o conteúdo era
fraudulento. Em nota oficial, Honjo negou:

\begin{quote}
Fico muito triste por meu nome e o da Universidade de Kyoto terem sido
usados ​​para espalhar falsas acusações e desinformação. Este é um
momento para que todos nós, especialmente aqueles que dedicam suas
carreiras aos pioneiros da pesquisa científica, trabalhemos juntos para
combater esse inimigo comum.\footnote{Luiz Fernando Menezes. Nobel de Medicina não disse que novo coronavírus foi criado pela China. \textit{Aos Fatos}, 28 de abril de 2020.}
\end{quote}

Essa postura traria sérias consequências para as relações comerciais e
diplomáticas entre Brasil e China, principal destino das exportações
brasileiras de produtos manufaturados e agropecuários, segundo o
Indicador de Comércio Exterior (\textsc{icomex}) da \textsc{fgv}. No entanto, essa não parecia ser a preocupação do presidente e de
seu círculo de apoiadores mais próximos.

Foi com indignação que a Embaixada da China no Brasil repudiou o
posicionamento de Eduardo Bolsonaro e enviou a seguinte nota: ``As suas
palavras são extremamente irresponsáveis e nos soam familiares'',
respondeu em sua conta oficial de Twitter o embaixador da China no
Brasil, Yang Wanming. O diplomata afirmou que a postura anti-China não
seria aceitável a uma figura pública.

Quanto mais as negociações entre o governo do estado de São Paulo e o
laboratório chinês avançavam, mais o governo federal se dedicava a negar
a importância da vacinação, condenar a importância do isolamento social,
criticar a obrigatoriedade do uso de máscaras em lugares públicos e
convocar a população para se manifestar contra as medidas adotadas por
prefeitos e governadores.

Essa postura negacionista em relação à vacinação era inédita no Brasil,
estava na contramão do histórico do país. O Programa Nacional de
Imunização (\textsc{pni}), criado em 1973, tinha o objetivo de garantir a todos os
cidadãos o acesso às vacinas e que a vacinação maciça fosse capaz de
erradicar determinadas doenças. Com experiências bem-sucedidas, o Brasil
entre os anos 1960 e 1970 participou da Campanha Mundial de Erradicação
da Varíola. Sempre à frente em campanhas de imunização, obteve
resultados surpreendentes, diminuindo o número de mortes de várias
doenças.

O governo de Bolsonaro virava essa página da história da saúde
brasileira. A preocupação de infectologistas era que o ambiente de
desconfiança fizesse com que a população não aderisse à vacinação, o que
dificultaria ainda mais a imunização coletiva. Bolsonaro recusou a
oferta de insumos de laboratórios internacionais importantes,
inicialmente, criando nas suas bases digitais o discurso contrário --
primeiro à vacina chinesa, e posteriormente levantando suspeitas a
qualquer outra.

A desinformação institucional não parava e Bolsonaro emperrava a
negociação da compra de vacinas, sob o pretexto de discordar de diversas
cláusulas contratuais apresentadas pelo laboratório que comercializava o
imunizante. Chegou a dizer que era muito preocupante que a vacina
mexesse com o sistema imunológico das pessoas e que aqueles que a
tomassem poderiam sofrer mutações e até mesmo virarem jacaré. Ou,
demonstrando ainda mais sua misoginia e homofobia, dizia que poderia
começar a aparecer barba nas mulheres e que a vacina poderia fazer com
que os homens começassem a falar fino.

Em dezembro de 2020, o Reino Unido deu o pontapé inicial à vacinação.
Foi o primeiro país a usar o imunizante. A vacina passava do plano da
pesquisa e se materializava para a população global. Ironicamente, o
primeiro-ministro Boris Johnson, que no início da pandemia adotou um
discurso enfaticamente negacionista, após contrair o vírus e ser
hospitalizado arrefeceu essa narrativa, dando início à organização da
imunização no seu país.

Falar contra a vacina enquanto ela não existia era uma coisa, mas quando
passamos a ver as cenas de países que iniciavam suas campanhas de
imunização, a opinião pública começou a mudar. Evidentemente que os
países que conseguiram vacinar suas populações rapidamente são os mais
ricos e que o caminho seria longo para alcançar percentuais aceitáveis
de vacinação em âmbito global.

Era impossível ignorar a desigualdade social no combate à pandemia e
isso reflete nas condições precárias em que vive parte expressiva da
população mundial, fatores que incidem diretamente nas formas de
contágio e de recuperação dos infectados. Os boletins epidemiológicos
divulgados pelo Ministério da Saúde traziam informações de raça e cor
nas internações e mortes pelo novo coronavírus, confirmando que morrem
40\% mais negros que brancos, aproximadamente. Dados que denunciam a
situação dramática da população pobre, preta e periférica, abandonada à
própria sorte.

Finalmente, em janeiro de 2021, a vacinação era iniciada no Brasil. Mas
ao mesmo tempo que trazia alguma esperança, também mostrava o quanto
faltava para a imunização da população num país que não tinha
planejamento e, como veremos mais à frente, não havia comprado as
vacinas, ignorando oferta de laboratórios interessados em negociar a
venda do imunizante.

Nesse momento o Brasil começou a viver a chamada segunda onda da
pandemia, que continuava a lotar as \textsc{uti}s. A cidade de Manaus atravessou
uma crise grave de falta de oxigênio, provocando uma situação crítica
que expôs ainda mais o então ministro da Saúde, general Eduardo
Pazuello, que demorava em dar respostas efetivas ao morticínio, tornando
sua permanência no ministério cada dia mais insustentável.

\section{A descoberta do gabinete paralelo na \textsc{cpi}}

A ausência do Plano Nacional de Imunização para fazer a coordenação da
vacinação, as atitudes de boicote explícito do Ministério da Saúde, a
falta de oxigênio e insumos para a produção de vacinas causavam
movimentação no Congresso Nacional, que conseguiu aprovar a criação de
uma Comissão Parlamentar de Inquérito (\textsc{cpi}) para investigar as ações e
omissões do governo federal no enfrentamento da pandemia. Depoimentos
revelavam aspectos ainda obscuros da atuação governamental, além de
demonstrar uma rede conspiratória dentro das instituições, utilizando-se
das estruturas oficiais para boicotar a vacinação.

Personagens sem grande expressão pública começam a aparecer e a mostrar
parte do maior esquema de aparelhamento institucional, responsável pela
produção desinformativa que teve como decorrência milhares de mortes.

O presidente era assessorado para a tomada de decisões por técnicos que
não faziam parte do Ministério da Saúde, mas integrantes de uma espécie
de gabinete paralelo, composto por profissionais adeptos do negacionismo
científico. Esse grupo foi criado para o aconselhamento de decisões em
relação à pandemia e produzia as bases para que fossem elaboradas a
desinformação compartilhada nas mídias sociais. Embora não fizessem
parte da estrutura institucional, verifica-se a materialidade dessa
consultoria em diversos atos e medidas oficiais.

A intenção verificada na ação sistemática do governo federal revela uma
atuação orquestrada para disseminar o vírus em âmbito nacional para
atingir a imunidade de rebanho pelo alto índice de contágio da
população --- medida, como dissemos anteriormente, não recomendada por
nenhuma organização sanitária internacional. Não é simples demonstrar
quantas mortes poderiam ter sido evitadas, caso houvesse uma conduta
adequada e coordenada por parte das autoridades públicas. Mas é possível
confirmar que a política governamental articulava a produção de
informações falsas, mentiras e teorias da conspiração com atos oficiais
que provocaram situações de vulnerabilidade da população em relação ao
contato com o novo coronavírus.

A partir da análise minuciosa de portarias, medidas provisórias,
leis, resoluções e decretos governamentais articulados às declarações
públicas do presidente, há clareza da intencionalidade. É o que ficou
demonstrado no boletim \textit{Direitos na pandemia --- mapeamento e análise das
normas jurídicas de resposta à \textsc{covid-19} no Brasil}, desenvolvido
conjuntamente pelo Centro de Pesquisas e Estudos de Direito Sanitário
(Cepedisa) da Faculdade de Saúde Pública (\textsc{fsp}) da Universidade de São
Paulo (\textsc{usp}) e a Conectas Direitos Humanos.

O esforço dos pesquisadores mapeou 3.049 normas relacionadas à \textsc{covid-19}
somente em 2020. Tratava-se de uma ação direta entre os atos normativos
federais buscando obstruir as iniciativas locais, além da propaganda
sistemática contra a saúde pública por parte do governo. Exemplo desse
tipo de norma jurídica está na insistência em impedir a obrigatoriedade
do uso de máscara em locais públicos, decreto adotado por vários
prefeitos e governadores que Bolsonaro tentava barrar com outro decreto.
Foi impedido pelo \textsc{stf}, que garantiu aos poderes locais a liberdade para
o estabelecimento das medidas consideradas necessárias para impedir o
avanço do número de infectados.

Essas iniciativas oficiais mostram não somente a negligência do governo
federal, a inoperância do Ministério da Saúde, mas principalmente
reiteram seu caráter genocida. Aquilo que parecia amadorismo ou
incompetência ficava cada vez mais claro que se tratava de uma
estratégia, uma estratégia de morte, uma estratégia institucional de
propagação do vírus.

As informações sistematizadas no Boletim comprovam o uso das
instituições para materializar a prática antissistêmica, dentro do
próprio sistema, ou seja, demolir as instituições por dentro e aqui vale
relembrar a orientação de Steve Bannon, estrategista da \textit{alt-right}
norte-americana, ex-assessor de Donald Trump que mantém vínculos estreitos com
a família Bolsonaro. Trata-se não somente da produção e disseminação em
massa de desinformação, mas da \textit{efetivação da ultranecropolítica}. Poderíamos
afirmar, com isso, que o contexto pandêmico contribuiu para a implementação
acelerada de uma política genocida.

No caso da vacinação, temos uma das demonstrações mais concretas da
política de morte adotada pelo governo federal através da narrativa
negacionista composta pelos seguintes argumentos: minimização da
pandemia (``é só uma gripezinha''), questionamento do elevado número de
mortes, defesa da imunidade de rebanho por contágio, questionamento das
medidas de isolamento social através da promoção de aglomerações, não
adoção do uso de máscaras e, ainda mais grave, a não negociação da compra de
vacina em tempo hábil para garantir a imunização da populações.

A falta de empatia do presidente em relação às famílias e amigos das
vítimas causava espanto até mesmo em alguns de seus eleitores, que demonstravam
arrependimento ou decepção. As pesquisas de
opinião pública revelavam a queda da popularidade e, de certa forma, um
certo enfraquecimento da narrativa negacionista.

Neste cenário, a narrativa antivacina por parte de Bolsonaro começava a
ser abalada quanto mais se divulgavam as imagens da imunização em vários
países do mundo, fazendo com que as pessoas começassem a mudar de
opinião. Segundo pesquisa realizada pelo Datafolha em maio de 2021, 91\%
dos brasileiros pretendiam se vacinar e foi o que aconteceu quando as
unidades de saúde abriram a vacinação. A desconfiança disseminada pelas
redes bolsonaristas em relação aos imunizantes vão dando lugar à adesão
popular, que começa a criticar a demora no processo e a
demonstrar descrédito na narrativa presidencial.

\chapter[O uso das redes digitais e o impacto na ordem\\democrática]{O uso das redes digitais e o impacto\\na ordem democrática}

Em diversas partes do mundo, a democracia vem sendo questionada e parte
da literatura da ciência política recente tem se dedicado a compreender
os motivos que a levam a morrer ou a chegar ao fim. Esse
diagnóstico é fundamental para tentarmos enfrentar as sucessivas crises
dos Estados democráticos que abrem caminho para as ideologias
autoritárias ganharem força no tecido social.

Para compreendermos o contexto político brasileiro precisamos recuperar
alguns aspectos históricos que possibilitaram a eleição de um governante
com o perfil ideológico de extrema-direita. A vitória de Jair Bolsonaro
em 2018 causou espanto nos setores democráticos e progressistas que não
conseguiam compreender como havia sido possível aquele resultado. Após
mais de 30 anos da redemocratização, se acreditava que o espírito dos
tempos autoritários teria ficado no passado.

%ñ está na bibliografia: avritzer  (2018)
Alguns analistas apostavam na curta duração desse período, outros que
ele seria mais longo do que prevíamos. Avritzer sinalizou para o 
caráter pendular da democracia brasileira e como encontramos na nossa
institucionalidade a presença tanto das estruturas democráticas quanto
das não democráticas que podem ser acionadas de tempo em tempo. A partir
de junho 2013, com os protestos que levaram milhares de manifestantes às
ruas, inicialmente com pautas de mobilidade urbana e posteriormente com
a ampliação e incorporação de reivindicações difusas, observou-se a
entrada de novos atores políticos que estavam longe dos espaços
públicos.

Começamos a observar o crescimento de forças políticas com a agenda
vinculada aos princípios conservadores e também de extrema-direita e,
além de entender que a consolidação democrática ainda estava distante,
notava-se uma regressão em alguns valores e costumes que passaram a
resultar em aumento de conflitos entre grupos políticos. Naquele
contexto, começava a ganhar destaque maior os protagonistas que sempre
estiveram na cena política, mas não chegavam a ocupar a centralidade e
tinham aparições mais episódicas.

Um conjunto de fatores básicos são necessários para a composição do
ambiente social e político para que o pensamento autoritário consiga uma
capilaridade ao ponto de ser incorporado e legitimado por parte
expressiva da população. A produção discursiva passa a ter um papel
relevante e o uso de estratégias para alcançar os objetivos dessas
forças políticas ganham ingredientes cada vez mais sofisticados,
incorporando o uso de dinâmicas comunicacionais, associada aos
dispositivos tecnológicos e informacionais. As lideranças políticas
autoritárias apostam na criação de polarizações forjadas
artificialmente, adotando discursos e práticas no mínimo questionáveis e
até mesmo criminalizáveis. A elaboração de narrativas convincentes é
fundamental, mesmo que com argumentos duvidosos, sem fundamentos, sem
embasamento nos fatos, mas acionando a lógica do \textit{nós contra eles}, o bom
e velho inimigo a ser combatido ou eliminado.

Em outros tempos era incomum ouvirmos na esfera pública a apologia ao
nazismo e seus crimes. O enaltecimento das ideias nazistas ficava
circunscrito aos grupos minoritários, ultra-extremistas, sem expressão
na sociedade e rapidamente eram questionados por setores progressistas
com argumentos e registros que mostravam os horrores promovidos pelo
regime que eliminou milhões de pessoas na primeira metade do século \textsc{xx}.

Há mais de 50 anos, o filósofo alemão Theodor Adorno afirmava que a
educação após o Holocausto só teria sentido para evitar a repetição
daquela barbárie --- referindo-se a importância da educação para evitar
situações vividas durante o nazismo. A atualidade desse alerta ocorre
justamente num momento em que um certo revisionismo histórico abriu
caminho não somente para totalitarismos analisados pelo autor, mas para
as práticas autoritárias crescentes em sociedades democráticas. A
abordagem de Adorno nos ensina que as lideranças autoritárias buscam
investir sistematicamente no potencial antidemocrático, potencializando
medos enraizados nas pessoas para torná-las predispostas a certas
crenças e resistentes aos fundamentos básicos da democracia.

Nos anos 1940, Adorno coordenou uma pesquisa nos \textsc{eua}, com a realização
de entrevistas para verificar se seria possível que um fenômeno como o
nazifascismo alemão ocorresse em um país com uma dinâmica democrática
consolidada. Os resultados apontaram questões fundamentais para se
conhecer os microfascismos e compreender como as pessoas podem se tornar
mais suscetíveis às influências da propaganda ideológica autoritária.

Alimentar ressentimentos, preconceitos, visões negacionistas ou
revisionistas constitui parte da estratégia desse tipo de líder com o
objetivo de preservar e ampliar seu leque de apoiadores. Esse tipo de
atuação não é novo e foi adotado pelas principais lideranças
autoritárias ao longo da história. Nesse sentido, a reflexão
desenvolvida por Adorno, além da atualidade, contribui para compreender a
presença do fascismo potencial ou da dinâmica da personalidade
autoritária mesmo nas sociedades democráticas. Para o autor, na Alemanha
nazista, as pessoas não necessariamente se identificavam com a ideologia
do regime, porém, em alguma medida, expressavam princípios conservadores
e antidemocráticos que eram alimentados pelos governantes para manter a
população em sintonia com esses valores.

Em outra perspectiva, mas corroborando com essa reflexão, encontramos o
instigante ensaio da filósofa alemã, Hannah Arendt, intitulado ``Verdade e
Política'', publicado em 1967. Logo no início, apresenta as seguintes
indagações:

\begin{quote}
As mentiras foram sempre consideradas como instrumentos necessários e
legítimos, não apenas na profissão de político ou demagogo, mas também
na de homem de Estado. Por que será assim? E o que é que isso significa
no que se refere à natureza e à dignidade do domínio político, por um
lado, e à natureza e à dignidade da verdade e da boa-fé, por outro? Será
da própria essência da verdade ser impotente e da própria essência do
poder enganar?\footnote{Hannah Arendt. \textit{Entre o passado e o futuro}. São Paulo: Editora Perspectiva, 1997.}
\end{quote}

A filósofa estava preocupada em discutir as verdades de fato e pensar
nos acontecimentos que são engendrados pelos homens em suas ações na
sociedade. Para ela, é a verdade de fato que interessa, embora reconheça
que pode ser efêmera e falsificada ao ponto de apagar ou alterar os
acontecimentos, considerando a fragilidade deles. Além dessa natureza
frágil, os fatos e as opiniões são facilmente confundidos, mesmo que
tenham origens diferentes e reconhecendo que cada momento histórico
possa escrever sua própria história, Arendt afirma que não se pode
permitir que os fatos sejam reconstituídos, apenas suas interpretações.

O Dia Internacional em Memória às Vítimas do Holocausto, comemorado
anualmente, relembra o momento em que as tropas da União Soviética
chegaram ao campo de concentração de Auschwitz. A libertação dos judeus
é um fato histórico e é considerada um símbolo do Holocausto. Mesmo que alguns 
partidários dos ideais nazistas insistam em negá-lo, é
importante lembrar que mais de seis milhões de judeus foram exterminados
pela Alemanha sob o domínio do regime nazista e por isso é essencial
abordarmos os negacionismos e revisionismos.

Os recentes episódios envolvendo a discussão sobre o Holocausto nos
fazem retomar um debate importante a respeito das motivações que levam
as pessoas a aderirem às informações fraudulentas e mentirosas sobre
ele. O crescimento desse tipo de narrativa é preocupante, tendo em vista
o aumento de posturas defensoras de retrocessos e práticas políticas
inspiradas nos totalitarismos.

O filme \textit{A negação},\footnote{Filme dirigido por Mick Jacson e
  escrito por David Hare, filmado no Reino Unido e nos Estados Unidos. Produzido pela \textsc{bbc} e lançado em 2016, tem 110 minutos.} baseado em uma história real, narra uma
disputa judicial entre um negacionista e uma historiadora do Holocausto.
Trata-se de obra interessante para demonstrar a coragem de verdade
expressa na figura da pesquisadora que não permitiu que os fatos
históricos fossem revistos para beneficiar as narrativas nazifascistas
que negam a existência de campos de concentração e aborda a memória,
conservação e imagem do Holocausto para que não haja mais nenhuma
possibilidade de se repetir esse tipo de crime.

É fato que as ideias dos negacionistas conturbam o debate,
principalmente quando seus autores afirmam estarem sendo cerceados no
direito à liberdade de expressão ao defenderem que o Holocausto nunca
existiu. É importante lembrarmos que não se trata de liberdade de
expressão, mas sim da defesa de um genocídio que é crime previsto em
legislações nacionais e nos fóruns internacionais.

O crescimento da extrema-direita conservadora no mundo chama a atenção
dos estudiosos do fenômeno. São movimentos, partidos, governantes que
estão se organizando em seus respectivos países, como podemos observar a
Frente Nacional Francesa (França), a Aurora Dourada (Grécia), a Pegida
(Alemanha), o Partido da Liberdade (Áustria), o Partido Lei e Justiça
(Polônia), a Liga Norte (Itália), o Vox (Espanha) além dos governantes
Viktor Orbán da Hungria, Donald Trump dos \textsc{eua},
Volodymyr Zelensky da Ucrânia, Recep Tayyip Erdogan da Turquia, Rodrigo Duterte
das Filipinas, Jeanine Áñez da Bolívia, e Jair
Bolsonaro do Brasil, para citar os maiores expoentes desse ideário. Ao
observamos essas lideranças políticas, os movimentos e partidos que
compõem esse campo ideológico da extrema-direita, podemos identificar a
desinformação e o negacionismo como um dos pontos em comum.

Alguns acontecimentos recentes da política brasileira apontam para a
necessidade de refletirmos sobre a importância de qualificarmos o debate
político com base em fatos e não em interpretações descontextualizadas
da realidade ou falsificadas por meio de um conjunto de estratégias
adotadas pelos grupos de extrema-direita, cuja crescimento global
preocupa cada vez mais aqueles que buscam manter princípios básicos de
uma sociedade democrática. Entre essas estratégias temos a prática
sistemática da desinformação.

A comunicação direta com a população é uma das formas empregadas para
que o ideário autoritário prospere e mantenha o maior número de pessoas
sintonizadas com os valores conservadores e até mesmo reacionários. Jair
Bolsonaro, ao longo de sua trajetória, construiu um discurso composto
por pautas anti-direitos humanos, defendeu a prática de tortura, o uso
da violência, o extermínio de comunistas e a defesa da ditadura militar.

Com isso, se estabeleceu uma imagem pública da salvação nacional contra os
inimigos do país, os antipatriotas, os comunistas, reforçando a ideia do
mito salvador que foi essencial para configurar uma base eleitoral fiel
e consolidar um movimento: o bolsonarismo. Essa demonstração é típica do
culto à personalidade, da mitificação e glorificação do líder
político, aspectos característicos de lideranças autoritárias. Não é por
acaso que seus seguidores mais radicais se referem a ele como \textit{o
mito}.

A desinformação, o medo, o discurso de ódio, o racismo, a homofobia e
diversas manifestações de intolerância vêm sendo utilizadas para
influenciar eleições e processos políticos através de uma lógica de
engajamento que se diferencia das formas tradicionais pelo uso intenso
das redes digitais para impulsionar uma dinâmica comunicacional jamais
vista anteriormente. As táticas de compartilhamento de desinformação por
Facebook, WhatsApp, Telegram, Instagram, YouTube, entre outras
plataformas, encontram eco nas crenças conservadoras presentes no campo
social. Cada postagem é calculada para gerar o engajamento, mesmo que
haja incoerência ou mentira na mensagem divulgada, demonstrando a falta
de compromisso ético com os efeitos daquilo que se dissemina.

Os partidos de extrema-direita são os diretos beneficiários da
propagação de emoções como raiva, ódio e ressentimento impondo uma
conduta de desconfiança em tudo o que não seja eles mesmos. A exemplo
disso, temos inúmeras declarações e ações de Donald Trump e Jair
Bolsonaro, apenas para mencionar dois dos representantes proeminentes
dessa perspectiva política.

No início dos anos 2000, a Internet enchia de esperança o campo
progressista por ser uma rede de redes e possibilitar a abertura às
vozes sempre silenciadas. Isso de fato ocorreu e abriu brechas para a
articulação de indivíduos e grupos que puderam potencializar suas formas
de resistência. No entanto, não estava tão presente nos debates,
influenciado por um tipo de \textit{ciberotimismo}, que a rede também abriria
oportunidades para discursos de ódio e desinformação.

Alguns pesquisadores, mesmo que timidamente, fizeram esse alerta, talvez
não o suficiente. A tendência crescente de ocupação da rede pela
desinformação apresenta métodos e estratégias bastante sofisticados. A
seguir abordaremos as técnicas e tecnologias que passaram a fazer parte
das estratégias das lideranças políticas de extrema-direita para
chegarem, permanecerem e corroerem as instituições democráticas.

Abordaremos brevemente a sofisticada apropriação tecnológica no processo
de modulação da opinião e de comportamentos políticos.

\section{Da técnica à produção da desinformação}

O ano de 2016 pode ser considerado um divisor de águas nas campanhas
políticas. Naquele momento foi realizado o plebiscito sobre a
permanência do Reino Unido na União Europeia, o Brexit. Em novembro do
mesmo ano, os norte-americanos também elegeram Donald Trump para a Presidência
dos Estados Unidos. Ambos os processos eleitorais têm em comum uma
forma inovadora de campanha para conquistar os eleitores. Além disso,
também tiveram em comum a empresa responsável pelo desenvolvimento dessa
estratégia: a Cambridge Analytica.

Frequentemente lemos menções sobre os escândalos da Cambridge Analytica
e acreditamos ser sumamente importante conhecer em detalhes as operações
realizadas pela empresa para compreendermos as conexões em torno dela.
Atores políticos, quase sempre vinculados ao ideário da extrema-direita,
articulados a bilionários influentes buscavam ampliar a influência em
processos políticos em diferentes países.

Cambridge Analytica, empresa que atuou na área de mineração e análise de
dados foi criada em 2013 e desde então atuou em aproximadamente 40
campanhas eleitorais em diversos países. Associada à Strategic
Communication Laboratories (\textsc{scl} Group), empresa britânica atuante em
pesquisas de comportamento e estratégia de comunicação, ambas tornaram-se
conhecidas por escândalos em processos políticos e eleitorais.

A jornalista do inglês The Guardian, Carole Cadwalladr, realizou
uma investigação minuciosa e consistente sobre as atividades da empresa
que oferecia um marketing político inovador com base na
microssegmentação de dados para avaliar a personalidade dos eleitores, a
partir das chamadas \textit{pegadas digitais} que são informações
disponibilizadas pelos usuários em diversas atividades realizadas na
Internet. Por meio de informações diferentes, como por exemplo, dados de
compras, cartão de fidelidade, associações a clubes, igrejas
frequentadas, livros e revistas pesquisados, entre outras, foi possível
construir um perfil desses usuários.

Entre as revelações trazidas por Cadwallard, temos a confirmação da
conexão direta entre os apoiadores da campanha favorável ao Brexit, e a
campanha de Donald Trump, mas esse era só o começo das articulações
entre ambos. Nigel Farage, um dos principais defensores e articuladores
da campanha favorável à saída do Reino Unido da União Europeia,
comemorou de forma entusiasmada o resultado da eleição norte-americana e teve
relações estreitas com Steve Bannon, coordenador da campanha de Donald
Trump que na época também era vice-presidente da Cambridge Analytica.
Farage faz parte de um grupo de bilionários que apoiou o Brexit e viram
na estratégia de campanha desenvolvida pela Cambridge Analytica a chave
para o sucesso das eleições norte-americanas.

Alexander Nix, \textsc{ceo} da Cambridge Analytica, desempenhou um papel
fundamental nas eleições norte-americanas de 2016 e pode ser considerado um
dos grandes mentores do sofisticado uso de dados em processos eleitorais.
Ganhou destaque nas prévias republicanas e já possuiu toda a estrutura
de dados sobre os eleitores norte-americanos, obtidos por meio de enquete e
conseguiu criar cinco mil pontos de medição que possibilitavam prever a
personalidade de cada eleitor, portanto, poderia influenciar o
comportamento nas eleições. Foi contratado a trabalhar para a
candidatura de Donald Trump e adotou a estratégia de utilizar o estudo
do comportamento eleitoral para disseminar vídeos que atingissem a
reputação da adversária de Trump, a democrata Hillary Clinton. Os vídeos
associavam a imagem de Clinton à corrupção, escândalos sexuais
envolvendo crianças e outras mentiras.

Entre os vencedores das eleições norte-americanas, alguns nomes se destacam
pelo papel desempenhado na vitória de Donald Trump. Robert Mercer, um
dos donos da Cambridge Analytica, bilionário norte-americano que gerenciava
fundos de investimento, foi programador da \textsc{ibm} e seus amplos
conhecimentos de matemática fizeram com que ele descobrisse algoritmos
capazes de monitorar o mercado e, assim, enriquecer. É importante
destacar que Mercer foi apoiador de causas identificadas com o perfil
ultraconservador e foi por isso que se aproximou de Steve Bannon, um dos
diretores da empresa.

A \textit{alt-right} norte-americana, também conhecida como direita alternativa,
articula-se em torno da rejeição ao conservadorismo clássico e se
organiza em torno de sexismo, antissemitismo e xenofobia. São adeptos e
produtores de teorias da conspiração e apoiadores dos supremacistas
brancos, grupos com ideias claramente racistas. As origens dessa extrema-direita podem ser localizadas em fóruns da Internet, como por exemplo
\textit{4chan} e \textit{8chan}, uma espécie de fábrica de produção de \textit{memes} utilizadas
pelos seguidores desse ideário ultraconservador.

A linguagem do deboche, da caricatura, da agressividade e do desrespeito
são características das práticas comunicativas desse ideário. Trata-se
de uma forma de afirmar que falam \textit{livremente} sobre qualquer tema e
que não podem ser censurados. Afirmam serem contrários ao chamado
\textit{politicamente correto} e é possível ver uma carga política bastante
intensa nesse tipo de linguagem amplamente utilizada nas mídias desse
espectro ideológico, principalmente em canais do YouTube.

Outro termo utilizado por esse grupo é o \textit{marxismo cultural}, também
criado pela extrema-direita que se inspirou na obra \textit{Minha luta},
de Adolf Hitler. Segundo essa teoria, o marxismo econômico dos anos 30
fracassou e esse espaço teria sido ocupado por judeus que começavam a
influenciar fortemente a cultura para destruir os valores conservadores.

É fundamental observarmos as conexões internacionais da extrema-direita
conservadora com o presidente Jair Bolsonaro. No discurso de posse
presidencial, ele disse que acabaria com o socialismo e o politicamente
correto, demonstrando claramente sua intenção em ativar a equação \textit{nós
contra eles}, ou forjar a lógica \textit{amigo versus inimigo}, tão utilizada
pelo populismo de direita para atacar os movimentos sociais, partidos
políticos e intelectuais afinados com o pensamento do campo progressista
e de esquerda. O termo \textit{politicamente correto} é muito vago,
impreciso, pode adquirir vários significados, mas frequentemente é
utilizado por aqueles que praticam diversas formas de intolerância,
buscando naturalizá-las, além de passar atestado de espontaneidade, algo
do tipo ``falo o que me vem à cabeça, não estou nem aí''.

A extrema-direita explora sistematicamente as dinâmicas emocionais para
mobilizar a população no sentido de ampliar sua força política e ganhar
a disputa de narrativa. Para tanto, adota práticas discursivas capazes
de atingir os afetos individuais e coletivos. Utiliza de técnicas para
compreender os sentimentos, a raiva, o ódio que passam a ser trabalhados
e apresentados sob a roupagem da autenticidade.

Não é novidade o uso de métodos para o monitoramento do comportamento
político e da opinião pública. A Strategic Communication Laboratories
(\textsc{slc}) atuava na análise de informações de indivíduos para classificar
seus comportamentos e eram especializados em Operações Psicológicas.\footnote{Psychological Operations (\textsc{psyop}) --- termo militar para definir o uso de técnicas da psicologia
para monitorar o comportamento individual e coletivo.} Trata-se de
dinâmicas para transmitir informações e indicadores selecionados a fim
de influenciar as emoções e reforçar o comportamento favorável aos
princípios e objetivos dos Estados para os quais trabalha. Existem três
tipos principais de atuação: estratégica, operacional e tática.

As \textsc{psyop} estratégicas incluem as atividades informativas conduzidas
pelas agências governamentais dos \textsc{eua} fora da arena militar. As
operacionais são conduzidas nas diversas operações militares em uma área
operacional específica para promover a eficácia das campanhas e
estratégias do comando. Já as táticas são conduzidas na área designada a
um comandante especializado em operações militares para apoiar uma
missão contra as forças adversárias.

É importante observarmos que o método pode ser utilizado --- e tem sido
com alguma frequência --- para estimular revoltas populares em países em
que o governo norte-americano tem interesses políticos ou econômicos. Busca
definir ações para provocar uma reação esperada em um indivíduo ou em
grupos, ou seja, a partir da definição de um determinado público-alvo
que se quer exercer a influência. Pode ser usado até mesmo para induzir
confissões ou reforçar atitudes e comportamentos favoráveis aos
objetivos de quem o aplica, além de ser muito útil para destruir
moralmente os adversários.

O uso dessas técnicas também não é exatamente uma novidade, mas o que
vimos foi a introdução de novas formas de monitoramento de redes
digitais atuando nesse campo. O uso do \textit{Big Data} pode ser observado a
partir das revelações dos procedimentos ilegais realizados pela
Cambridge Analytica em 2016. A partir das denúncias de Christopher
Wylie, ex-programador da empresa, foi possível conhecer alguns detalhes
sobre os bastidores, os métodos e os envolvidos, destacando-se entre
eles Steve Bannon. As revelações produziram um impacto midiático global
e afetou políticos e instituições britânicas e norte-americanas.

De acordo com Wylie, o propósito da empresa não era somente atuar na
área de ciência de dados, mas principalmente de propaganda. As inovações
estavam baseadas em modelos matemáticos para elaboração da publicidade,
utilizando técnicas de modelagem de dados, mineração massiva de
informações e categorias psicológicas. A partir da combinação dessas
técnicas utilizadas para a análise comportamental era possível obter
respostas aos mais diversos estímulos, no caso específico, obter com o
máximo de precisão as possibilidades de assimilação de mensagens
políticas por parte de eleitores.

O processamento dessas informações foi desenvolvido graças ao aplicativo
criado por Alexander Kogan, pesquisador da Universidade de Cambridge que
trabalhava no Centro de Psicometria de Cambridge, com os psicólogos
David Stillwell, Thore Graepel e Michal Kosinski. O laboratório era
responsável pelo desenvolvimento de pesquisas para compreender as
diferenças individuais em comportamentos e identificar traços de
personalidade a partir das características psicológicas dos indivíduos.
As pesquisas realizadas pela equipe se utilizavam de diversos métodos
computacionais, incluindo \textit{machine learning} (aprendizado de máquina), uso
dos algoritmos, mineração de dados e processos de observação com milhões
de participantes.

Empregando o método chamado \textit{Big Five} ou \textsc{ocean},\footnote{\textsc{ocean}
  é um acrônimo em inglês, composto pelas seguintes palavras: \textit{Openness,~conscientiousness,~extraversion,~agreeableness}~e~\textit{neuroticism}.} 
a Cambridge Analytica utilizava as informações coletadas por Kogan para
criar medições de traços psicológicos e poder classificá-los a fim de criar segmentações e disseminar mensagens aos seguintes perfis:

\begin{enumerate}
%\def\labelenumi{\arabic{enumi})}
\item Abertura (\textit{a novas experiências})
\item Consenciosidade (\textit{perfeccionismo})
\item Extroversão (\textit{sociabilidade})
\item Condescendência (\textit{cooperatividade})
\item Neuroticismo (\textit{temperamento})
\end{enumerate}

O método permitia saber se as pessoas eram impulsivas ou tranquilas,
extrovertidas ou tímidas, controladas ou explosivas e, assim por diante.
Era evidente compreender o interesse da empresa pelas pesquisas de
psicometria e neurociência, considerando que visava com esse
conhecimento atingir os usuários das redes sociais. Desse modo, foram
adotadas as técnicas de modulação de mensagens desenvolvidas para
influenciar os eleitores no plebiscito da Inglaterra e na campanha
eleitoral norte-americana, tendo favorecido as posições políticas dos
conservadores.

Significa dizer que os indivíduos recebiam as mensagens que se
enquadravam com alguns valores e crenças relacionadas ao seu perfil de
personalidade. A utilização do método do \textit{Big Five} favorecia uma forma de
reforçar as segmentações, as bolhas e os jardins murados, impedindo
práticas de diálogo, trocas e interações. Como consequência, temos a
falta de debates, o enclausuramento em grupos de convicções, trazendo um
impacto significativo para a democracia.

Esse comportamento produz uma espécie de câmera de eco ideológica,
considerando que as ideias ou crenças que circulam nessas redes são
amplificadas pela repetição de um mesmo padrão, de uma mesma visão de
mundo, portanto, os indivíduos ficavam vulneráveis a propagandas que, em
muitas situações, reproduziram somente desinformação.

\section{Da desinformação à ação governamental nas campanhas eleitorais}

Uma breve análise do contexto político em que se realizaram as eleições
presidenciais norte-americana e brasileira é fundamental para compreendermos
os aspectos que podem ser considerados decisivos para a chegada àqueles
resultados, e para isso é relevante observarmos não somente a dinâmica
política nacional, mas identificarmos algumas matrizes comuns de um
fenômeno global que mostra a expansão do ideário e os métodos da extrema-direita ultraconservadora.

Nas eleições norte-americanas de 2016, uma parcela significativa do eleitorado
estava desgastada após a crise financeira de 2008 que promoveu
empobrecimento brutal da população, principalmente das classes médias
que já viviam um processo de perda de empregos e renda pela dinâmica da
reestruturação produtiva. Como desdobramentos dessa crise verificamos o
desemprego elevado e a precarização do trabalho, para ficarmos apenas nos
temas do campo econômico.

Donald Trump conjugou alguns aspectos importantes, por exemplo, a
crítica ao globalismo, era apontada como responsável pelo fracasso da
economia norte-americana, fazendo coro com a \textit{alt-right}. Com a promessa da
retomada do crescimento econômico, a volta dos empregos e da América
para os norte-americanos, atraiu a atenção dos descontentes. Segundo Jeremy
Adelman, do Laboratório de História Global da Universidade de Princeton,
o fenômeno era chamado de \textit{nativismo}. Trata-se da construção de uma
narrativa em que os cidadãos norte-americanos teriam um lugar melhor na
sociedade, estimulando um discurso xenófobo, racista, em relação aos
imigrantes latino-americanos, além de reforçar o preconceito ao islã,
frequentemente associado ao terrorismo.

Dirigindo seu discurso para a classe média branca moradora de cidades
pequenas, com baixa escolaridade, perfil mais conservador, que sentiam-se
perdedores após a crise de 2008, período em que haviam ficado sem
empregos e que a bolha imobiliária tinha levado suas casas e as
economias de uma vida. O objetivo trumpista era claro: \textit{fazer a América
grande} para os norte-americanos, e os legítimos representantes eram os brancos
da classe média.

Donald Trump, com um estilo imponente e grande facilidade de comunicação
adquirida em seu famoso reality show, entrou na disputa das prévias do
Partido Republicano sem grandes chances, mas os norte-americanos --- e o mundo
-- foram assistindo a ascensão de um candidato que esbanjava narcisismo,
megalomania, culto à ignorância, enaltecendo os preconceitos raciais,
sexuais e de classe. Tudo isso na embalagem de alternativa aos políticos
considerados tradicionais e a ênfase no fato de ser um empresário de
sucesso e não um político de carreira. Uma candidatura assentada no
antissistema, característica marcante da extrema-direita em diversas
partes.

Além do papel da Cambridge Analytica abordado anteriormente, observamos
também o processo de produção industrial de desinformação realizado na
Macedônia, pequeno país localizado na península dos Balcãs, no sudoeste
da Europa. A cidade de Veles ficou conhecida como uma espécie de
``fábrica mundial de \textit{fake news}'', embora não fosse o único
celeiro desse tipo de produção. Durante as eleições norte-americanas de 2016,
aproximadamente cem sites em defesa da candidatura do republicano Donald
Trump estavam registrados no país pelos chamados \textit{Veles boys}, grupos de
jovens produtores de desinformação e notícias fraudulentas.

A partir do site Breitbart, que na época tinha Steve Bannon como um dos
proprietários, notava-se a ampliação na disseminação de desinformação e
mentiras produzidas principalmente em Veles e é importante ressaltar que
essas mentiras chamaram mais atenção e despertaram mais interesse do que
as manchetes veiculadas pelos grandes meios de comunicação. Silverman\footnote{This Analysis Shows How Fake Election News
Stories Outperformed Real News On Facebook. \textit{Buzzfeed News}, 16 de novembro de 2016.}
destaca que as vinte notícias fraudulentas com melhor performance
na rede social geraram 8.711.000 partilhas, reações e comentários,
número bastante expressivo que corrobora com a pesquisa realizada em
maio de 2016 pelo Pew Research Center, mostrando que 62\% dos adultos
norte-americanos consultavam notícias por meio das redes sociais. Em novembro
de 2016, mostrava que 79\% dos estadunidenses adultos entrevistados
declararam usar o Facebook, mais que o dobro dos utilizadores de outras
redes como Instagram 32\%, Pinterest 31\%, Linkedin 29\% e Twitter 24\%.

Algumas semelhanças podem ser identificadas nos contextos das eleições
presidenciais norte-americana, em 2016, e brasileira, em 2018. Não se trata aqui
de um estudo comparado, que exigiria uma análise mais aprofundada, mas de
destacar alguns aspectos que mostram além das analogias, elos entre
elas, principalmente ao identificarmos a construção de um movimento de
ultradireita em âmbito global. Coincidentemente, \textit{Movimento} é o nome dado
por Steve Bannon à articulação de grupos conservadores que atuam em
vários países. Uma marca característica desses grupos é a produção de
desinformação para influenciar a configuração da opinião pública.

Nesse sentido, vemos que a candidatura de Jair Bolsonaro nas eleições de
2018 se inspirou nos métodos utilizados por Trump e contou com uma
espécie de consultoria de Bannon e de Olavo de Carvalho, conhecido como
guru da extrema-direita brasileira. Desde 2005, Carvalho morava nos \textsc{eua}
e atuava como uma espécie de \textit{eminência parda} da família Bolsonaro.
Conhecido polemista das redes, Carvalho foi apontado como um dos grandes
articuladores da nova direita, identificado por um estilo agressivo,
criador e disseminador de discursos de ódio, negacionismo e um dos
maiores produtores de teorias da conspiração. Bannon e Carvalho
desempenharam papel fundamental nos bastidores dessas campanhas
eleitorais.

A vitória da extrema-direita em 2018 colocou o Brasil num circuito de
países que estão vivenciando a retórica populista de direita e trouxe um
cenário de profunda preocupação sobre os rumos da democracia brasileira,
bastante impactada desde o golpe de 2016. O crescimento dessa
ideologia em diversos lugares vem chamando a atenção dos setores
progressistas, não somente pelo significado da expansão dessas ideias,
mas pela adesão de parcelas da população aos princípios
antidemocráticos.

O papel das estratégias de comunicação utilizadas nas plataformas
digitais pelos atores políticos afiliados a esses grupos tem demonstrado
a sofisticação da estratégia para transformar a narrativa conservadora
em algo orgânico nas redes, como se fosse uma expressão espontânea dos
anseios da população. Esse contexto nos permite compreender o papel que
a desinformação ocupou, tendo em vista que foi amplamente utilizada
durante todo o período eleitoral.

Grupos de extrema-direita utilizam a desinformação para disputar a
opinião pública e corroer os pilares básicos da sociedade democrática
promovendo a ascensão do ultraconservadorismo.\footnote{David Runciman. \textit{Como a democracia chega ao fim}. São Paulo: Todavia, 2018.} O mais
preocupante é que estarmos perdendo a democracia como um valor
fundamental para as sociedades. Segundo o Barômetro das Américas,
realizado pela Fundação Getúlio Vargas e o Ibope no Brasil, em conjunto
com a Universidade Vanderbilt dos Estados Unidos, 60\% dos brasileiros
acreditam que a democracia é a melhor forma de governo e ao mesmo tempo,
58\% estão insatisfeitos com a dinâmica democrática. A contradição não é
ruim e expressa o descontentamento com uma democracia que não contempla
os aspectos sociais, a participação política efetiva e se restringe na
maioria das vezes à formalidade dos processos eleitorais. De qualquer
forma, os dados são um sinal desse descontentamento que abre caminho para
narrativas que apresentam soluções mágicas e milagrosas.

Entre os aspectos que afetam a imagem da democracia destacamos a
desigualdade que não tem apenas na crise econômica um de seus pilares,
mas na forma desigual como as riquezas são apropriadas no nosso país,
considerado um dos mais desiguais do mundo. Nosso abismo social expresso
nos indicadores socioeconômicos nos informa muito sobre essa triste
realidade. Destacamos esse ponto como fundamental por considerá-lo um
dos aspectos mais complexos para entendermos algumas narrativas que
embalam o debate político brasileiro.

Na última década verificamos que milhões de pessoas foram às ruas
protestar contra a corrupção, considerada o problema mais grave de nossa
história. Certamente, a corrupção é um dos grandes problemas em vários
países e um dos elementos que corroem as democracias, devendo ser
enfrentado com transparência, medidas de prevenção e de punição, entre
outras formas.

Por outro lado, é importante pensar na composição das manifestações
contra a corrupção, principalmente entre 2015 e 2016, com a presença
majoritária das classes médias urbanas e de setores da elite brasileira.
Não menos importante é também observarmos o perfil racial desses
protestos, cujas imagens mostram a maioria branca e a quase ausência de
negros e pardos que compõem parcelas expressivas da população. Uma das
imagens mais elucidativas desse contraste foi a de um casal branco,
vestido de verde amarelo, que levava ao protesto seus dois filhos
pequenos sob os cuidados de uma babá negra, vestindo uniforme branco,
atualizando a lógica da relação entre a \textit{casa grande} e a \textit{senzala},
descrita por Gilberto Freyre.

A pergunta que fica no ar é: por que um país com tanta desigualdade
social mobiliza setores médios e as elites para lutar contra a
corrupção, mas não faz o mesmo em relação ao abismo social histórico e
ao racismo estrutural? Consideramos que a expectativa de ampliação de
privilégios é um dos motivos que mobiliza as classes médias brasileiras
e norte-americanas que enxergam latinos, negros, entre outras minorias, como
adversários.

Desde as eleições presidenciais de 2014, para não retrocedermos muito no
tempo, verificamos um ambiente de insatisfação política articulado por
um bombardeio midiático em torno das denúncias de corrupção. Esse
processo desencadeou no aumento de ressentimentos políticos históricos,
habilmente catalisados pela extrema-direita conservadora, que incorporou
o discurso antissistêmico para enfrentar ``tudo o que está aí''.

A desconfiança em relação às instituições também é outro ponto
característico e permite se verificar o quanto a população estava abalada por um
discurso de profunda crítica aos partidos políticos, ao parlamento e, de
certa forma, ao judiciário e à mídia. A extrema-direita soube trabalhar
com grande habilidade esse sentimento de revolta, e as experiências de
fora do Brasil contribuíram significativamente para a elaboração das
estratégias adotadas.

Ainda em 2014, temos dois fatos muito importantes que abriram caminho
para o aprofundamento da crise de confiança na política e nas
instituições democráticas. Um deles foi a Operação Lava Jato, criada
para investigar crimes de corrupção e gestão fraudulenta envolvendo a
Petrobras, uma das mais importantes empresas públicas brasileiras e
entre as cinco maiores do mundo no ramo petrolífero. Rapidamente a Lava
Jato ganhou os noticiários com atividades realizadas em conjunto com a
Polícia Federal ganhando a atenção da população.

O processo de espetacularização das investigações de combate à corrupção\footnote{Tathiana S. Chicarino \textit{et. al.} Como dois estudantes de 22 anos puseram de pé o Sleeping Giants Brasil. \textit{Aurora: revista de arte, mídia e política}, São Paulo, v.\,14, n.\,40, p.\,6--27, fevereiro--maio de 2021.}
renderam páginas e mais páginas de jornais e revistas,
horas de transmissões televisivas e radiofônicas, ocupando também as
redes sociais digitais e a \textit{blogosfera}. As investigações eram coordenadas
pelo então juiz Sergio Moro que não saia dos holofotes, passando a ser
tratado como herói que livraria o Brasil da imoralidade dos desvios de
recursos públicos. Havia uma espécie de catarse nacional e um
endeusamento de juízes e promotores que se apresentaram como os
verdadeiros salvadores da pátria, os patriotas que tirariam o país das
mãos dos corruptos.

Vale ressaltar que naquele momento o Partido dos Trabalhadores (\textsc{pt})
ocupava a Presidência da República por doze anos e era frequentemente
atacado por setores da elite, contrários às políticas sociais elaboradas
pelos governos petistas e é essencial reconhecer o papel dos grandes
meios de comunicação, empresas que incorporaram, além da narrativa contra
a corrupção, a associação das investigações às lideranças políticas
do partido e a estabilização desse como o maior escândalo de corrupção do
país.

Diversos estudos no campo da comunicação política demonstram a
construção e disseminação dessa narrativa como forma de criar um
ambiente de criminalização do \textsc{pt} e semear o
\textit{antipetismo}. Com ares de espetacularização, era possível observar
lideranças políticas do partido, como o ex-presidente Lula, inegavelmente o alvo
central das operações, ser nomeado ``chefe de quadrilha''.

Com o crescimento do discurso anticorrupção, observamos a construção de
uma narrativa para incriminar a ex-presidenta Dilma Rousseff. 
Verificamos entre 2015 e 2016 manifestações massivas, amplamente
convocadas e divulgadas pelas grandes empresas de comunicação criando
uma ambientação totalmente desfavorável não somente ao partido, mas a
outros partidos políticos progressistas e aos movimentos sociais
populares e de esquerda. O espírito patriótico ganhou força nas ruas e
começamos a ver as forças de centro-direita que compunham o governo de
Rousseff abandonando o governo, como costumam fazer quando o vento
começa a mudar os rumos no país.

Chamou a atenção um dos discursos mais radicalizados na votação do
Congresso Nacional que aprovou o impeachment: a do então deputado
federal Jair Bolsonaro. Evocou a memória do coronel Carlos Alberto
Brilhante Ustra, dizendo ``o pavor de Dilma Rousseff''. Ustra foi
torturador da ex-presidenta, presa entre 1970 e 1972. Após o afastamento
de Rousseff e a ascensão de Michel Temer (\textsc{mdb}), a agenda política do país
ganhava outras diretrizes, as reformas neoliberais passaram a ser
defendidas e articuladas pelo governo com apoio das elites econômicas e
midiáticas. A reforma trabalhista ganhou narrativa de criadora de novos
empregos e a oposição não conseguia desmontar esse discurso defendido
até mesmo por setores das classes populares que acreditavam que as
mudanças ampliariam a empregabilidade.

Mas ainda faltava uma peça fundamental para considerar o \textsc{pt} fora do jogo
e sem possibilidade de retorno ao governo federal: Lula deveria estar
fora das eleições de 2018, e mais uma vez o juiz Sergio Moro entra em
ação e consegue mandar Lula para a prisão, num processo judicial
questionado e chamado de \textit{law fare}.

Lula iniciou uma luta judicial para ser candidato, tendo sido impedido
pela justiça com base na Lei da Ficha Limpa, às vésperas do processo
eleitoral. A decisão do Tribunal Superior Eleitoral (\textsc{tse}) foi questionada por
juristas em âmbito internacional, mas não houve reversão da decisão.
Lula foi substituído por Fernando Haddad (\textsc{pt}) em um cenário político
dramático e a candidatura petista enfrentou a desinformação durante toda a
campanha.

Nesse contexto eleitoral conturbado, o então candidato, Jair Bolsonaro,
sofreu um atentado em atividade pública de campanha, levando uma facada
de um dos participantes do evento. Foi hospitalizado e passou por
cirurgia. O episódio causou grande polêmica e mudou os rumos da
campanha, tendo sido determinante no resultado. A campanha foi ágil em
tentar associar o autor do atentado, Adélio Bispo de Oliveira, ao
Partido dos Trabalhadores, disseminando amplamente nas redes
bolsonaristas que ele era filiado ao partido e mostrando uma foto
manipulada em que aparecia junto com o ex-presidente Lula. A informação
foi desmentida pelo \textsc{pt} e o \textsc{tse} afirmou que na
lista de filiados do partido não constava esse nome.

Bolsonaro ficou fora dos debates, não apresentou um plano de governo,
além de ter gerado comoção nacional por ter sofrido violência grave e
quase perder a vida. Foi nesse cenário conturbado que os brasileiros
foram às urnas em uma das eleições mais atípicas após a redemocratização.

A eleição de 2018 deve ser considerada paradigmática também do ponto de
vista da comunicação política, principalmente pela circulação de boatos
e notícias falsas nas plataformas digitais. O uso de mentiras faz parte
da atividade política ao longo da história, conforme abordamos
anteriormente a partir da reflexão de Hannah Arendt. No entanto, o
fenômeno da desinformação observado durante esse pleito contou com o
extraordinário volume de mensagens políticas antidemocráticas e discursos de
ódio que exploravam medos e crenças existentes na população, estratégia
frequentemente utilizada pelos ultraconservadores.

Uma das mentiras amplamente difundidas para atingir a candidatura de
Fernando Haddad foi o chamado ``kit gay'', apresentada pelos opositores
aos etistas como uma espécie de cartilha para induzir crianças e
adolescentes a praticar a homossexualidade. Essa foi uma das mentiras
mais compartilhadas principalmente nos grupos de WhatsApp, conforme
podemos ver:

\begin{quote}
\textit{Haddad: o candidato do \emph{kit gay}. As crianças de 8 anos terão aula de
homossexualidade nas escolas}: esse é o alerta da imagem que circulou
em grupos de WhatsApp, no dia 26 de outubro de 2018, durante a corrida
eleitoral presidencial brasileira, em que o então candidato à
Presidência Jair Bolsonaro aponta para uma suposta manchete de
jornal. A foto do então candidato Fernando Haddad se destacava no
alto à esquerda da manchete, fazendo um sinal de positivo. Tratava-se
claramente de uma montagem, com uma informação inverídica, que foi uma
das mais disseminadas no ecossistema do WhatsApp no período, e um dos
temas mais recorrentes na campanha bolsonarista.\footnote{Silvana L. Almeida \textit{et al.} WhatsApp: a desordem da informação na eleição presidencial brasileira de 2018, \textit{Anais \textsc{vii} Simpósio Internacional Lavits}, Salvador, junho de 2019, p.\,2.}
\end{quote}

``Kit gay'' foi o termo cunhado pela bancada evangélica no Congresso
Nacional para se referir ao projeto Escola sem Homofobia que fazia
parte do Programa Brasil sem Homofobia, criado em 2004. A proposta era
desenvolver a formação de educadores para combater preconceito contra a
população homossexual e não previa a elaboração e distribuição de nenhum
tipo de material aos estudantes. Esse exemplo demonstra claramente como
funciona a criação da desinformação. O ponto de partida é correto, o
debate existiu, mas não com as características apresentadas e nem com os
produtos compartilhados pelas redes sociais. A informação foi distorcida
para originar um tipo de impacto no público conservador que sempre
apresenta resistências para discutir temas como a homofobia.

O projeto nunca foi colocado em prática, causou inúmeras polêmicas e
ficou claro que a preocupação era associar à imagem de Haddad a
exploração sexual de jovens e crianças ou à pedofilia. Esse tipo de
narrativa faz parte da estratégia da extrema-direita em âmbito global,
Trump utilizou o mesmo tipo de argumento contra Hillary Clinton em 2016,
em um escândalo conhecido como \textit{pizzagate}.

Segundo a agência Aos Fatos, que realizou o trabalho de checagem de
informações durante as eleições de 2018, as notícias falsas foram
compartilhadas 3,84 milhões de vezes no período eleitoral, e destaca-se
que entre as informações falsas que mais obtiveram atenção nas redes
está a do ``kit gay'', demonstrando o quanto essa narrativa foi
utilizada por bolsonaristas para atacar a imagem da candidatura petista.\footnote{Barbara Libório e Ana R. Cunha. Notícias falsas foram
compartilhadas ao menos 3,84 milhões de vezes durante as eleições. \textit{Aos
Fatos}, 31 de outubro de 2018.}

A coligação O Povo Feliz de Novo da candidatura de Fernando Haddad
entrou com uma representação junto ao \textsc{tse}
para retirar a postagem do ar, considerando que se tratava de
desinformação. O \textsc{tse} proferiu a seguinte decisão judicial:

\begin{quote}
No dia 25 de setembro de 2018, às 23 horas e 16 minutos, a página representada
utilizou-se de seu sítio eletrônico para ofender o candidato Fernando
Haddad, bem como a coligação O Povo Feliz de Novo, a qual informa que o
candidato estaria distribuindo mamadeiras em creches, com o bico no
formato de um órgão genital masculino, no que o narrador sugere que
seria ``com a desculpa de combater a homofobia, parte integrante do \textit{kit
gay}, uma invenção de Haddad'', \textit{fake news}. Segue a degravação do trecho
impugnado: ``Ó aqui, vocês que vota no \textsc{pt}, essa aqui é a mamadeira
distribuída na creche. Ó a marca aqui. Tá vendo? Distribuída na creche
pro seu filho. Com a desculpa de combater a homofobia. Olha o bico como
é. Tá vendo? O \textsc{pt} e Haddad pregam isso pro seu filho. Seu filho de 5, 6
anos de idade, vai beber mamadeira na creche com isso aqui. Pra combater
a homofobia. Tem que votar em Bolsonaro, rapaz. Bolsonaro que é pra
fazer o filho da gente homem e mulher. O \textsc{pt} e Haddad, Lula, Dilma, só
quer isso aqui pros nossos filhos. Isso faz parte do kit gay, invenção
de Haddad, viu?''.\footnote{Tribunal Superior Eleitoral. \textit{Processo: 0601530-54.2018.6.00.0000}. Representação, 2018.}
\end{quote}

A decisão do \textsc{tse} não era respeitada e os grupos continuavam disseminando
a mentira. Nunca é demais lembrar a velha expressão de Goebels, ministro
das Comunicações de Hitler, quando afirmava que uma mentira dita mil
vezes se transforma em verdade. Muitas pessoas diziam ter visto o ``kit
gay'', afirmando ter recebido o material de uma amiga, vizinha ou
familiar, reafirmando que nunca foi elaborado.

O apelo aos valores conservadores, presentes na sociedade brasileira,
não pararam e foram ainda mais enfatizados. Considerando que o suposto
``kit gay'' era uma aberração sem precedentes nas campanhas eleitorais,
apoiadores bolsonaristas divulgaram um vídeo mostrando mamadeiras com
bico em formato de pênis que, supostamente, eram distribuídas nas
creches municipais quando Fernando Haddad era prefeito de São Paulo,
entre 2012 e 2016. No momento em que o vídeo foi publicado, a campanha
do \textsc{pt} estava em plena fase de transição da
candidatura e não soube responder com a mesma força esse nível de
ataque.

Foi nesse contexto que verificamos a intensificação das estratégias de
comunicação do comando da campanha bolsonarista para a propagação das
mensagens falsas pelas redes digitais com o objetivo de atrair novos
adeptos para a agenda política ultraconservadora. Cabe lembrar que a
candidatura de Bolsonaro tinha apenas oito segundos de televisão no
horário gratuito de propaganda eleitoral, portanto o uso da Internet era
fundamental.

O exíguo tempo do partido no \textsc{hgpe} foi compensado pela farta exposição
midiática em telejornais, emissoras de rádio e mídia impressa, que
realizou farta cobertura a respeito da recuperação da saúde de
Bolsonaro, após o atentado. Na cobertura da imprensa, sua presença era 
infinitamente superior à de qualquer outro candidato. Sem contar efetivamente 
com a dramaticidade da situação de ter um candidato à Presidência que havia 
sofrido um atentado em pleno processo eleitoral. Além da diversidade de teorias 
da conspiração sobre o assunto produzida por apoiadores bolsonaristas, as redes digitais
foram entupidas com mensagens falsas.

A desinformação nas redes sociais foi fundamental para favorecer a
candidatura da extrema-direita nesse cenário de profundas mudanças. Lula
fora da corrida presidencial, substituído por Fernando Haddad, Bolsonaro
fora dos debates, alegando estar se recuperando do atentado, mesmo que
tenha participado de atividades de campanha após sua saída do hospital.
É preciso reconhecer que Bolsonaro explorava essas redes pelo menos
desde 2011, portanto, era presença expressiva nessas plataformas bem
antes de 2018, o que facilitou a capilarização dessas atividades durante
a campanha.

Uma pesquisa publicada pelo Jornal Folha de S.\,Paulo em outubro de 2018
sobre eleições mostrou os hábitos da dieta informacional dos brasileiros
naquele momento. Revelou que dois em cada três eleitores brasileiros
tinham contas em redes sociais, o equivalente a 66\% do eleitorado.
Entre os eleitores jovens, esse índice atingiu 90\%, seguido por 55\% na
faixa de 45 a 59 anos e 32\% dos que estão acima de 60 anos.\footnote{Datafolha. \textit{Relatório Uso das redes sociais}. Instituto Datafolha: Eleições 2018.}

Considerando o caráter emblemático dessa campanha do ponto de vista da
Comunicação Política, podemos concluir que foram as eleições em que o
eleitorado demonstrou um índice expressivo de confiança nas informações
recebidas por meio das plataformas digitais. Segundo a mesma pesquisa,
aproximadamente 86\% dos eleitores acreditavam nas notícias
compartilhadas pelo WhatsApp, índice bastante elevado para o período.

\begin{quote}
Identificamos especificamente que existem elementos fortes sobre o uso
de instrumentos de automação para potencializar a distribuição de
informações entre diferentes grupos de WhatsApp. Também identificamos
que existe uma ação coordenada entre diferentes membros na atuação de
redes de grupos de discussão via WhatsApp.\footnote{Caio Machado e Marco Konopacki. \textit{Poder computacional: automação
no uso do whatsapp nas eleições --- estudo sobre o uso de ferramentas de
automação para o impulsionamento digital de campanhas políticas nas
eleições brasileiras de 2018}. Rio de Janeiro: Instituto de Tecnologia e Sociedade do Rio de Janeiro, 2019, p.\,21.}
\end{quote}

A campanha vitoriosa de Bolsonaro utilizou vários recursos das
plataformas digitais tanto para se comunicar com o eleitorado e potenciais
apoiadores, como para ampliar a sensação de engajamento na campanha.
Facebook, Twitter e WhatsApp foram as mais utilizadas pelo candidato que
disseminou conteúdos e mobilizou seus apoiadores. Vale lembrar que esse
amplo uso das plataformas foi mantido pelo presidente durante o mandato,
estratégia combinada às críticas aos grandes meios de comunicação para
deslegitimá-los, além de diversas agressões verbais a jornalistas.

\section{A desinformação como prática do governo}

A vitória de Bolsonaro começava a deixar mais claro que esse método de
comunicação permanente e sem intermediação com seus apoiadores não
ficaria apenas nos discursos de campanha, mas se constituiria em um dos
pilares que dariam sustentação às ações presidenciais. A partir da posse
em janeiro de 2019, o país passaria a ser governado com base na teoria da
conspiração, no discurso de ódio e na intolerância. Esse era apenas o
começo do processo de institucionalização da desinformação
governamental.

Para o êxito desse projeto de poder era necessário o empenho contínuo na
observação dos comportamentos dos apoiadores, coletando suas opiniões
para enquadrar nas decisões e declarações governamentais. Significa
dizer que ``todo dia é dia de eleição na campanha permanente''.\footnote{H. Heclo. Campaigning and governing: a conspectus. \textit{In}: \textsc{ornstein}, N.\,J. e \textsc{mann}, T.\,E. \textit{The Permanent Campaign and Its Future}. Washington:
American Enterprise Institute and The Brookings Institution, 2000.}

Antes mesmo de tomar posse, Bolsonaro cogitou nomear para a \textsc{secom} o próprio filho, 
Carlos Bolsonaro, apontado como
o estrategista da campanha nas redes sociais desde sempre. Mesmo sem ocupar um
cargo na estrutura do Palácio do Planalto, o número \textsc{ii}, como o presidente
gosta de chamar os filhos, sempre desempenhou um papel fundamental na
comunicação do governo e principalmente do presidente.

Carlos Bolsonaro chegou a ocupar uma sala ao lado do gabinete do pai, o
que facilitava a coordenação da ofensiva comunicativa do governo. O
vereador pelo estado do Rio de Janeiro não se preocupava com esse
\textit{desvio} de função. Não mostrava nenhum constrangimento em não cumprir
com as obrigações de seu mandato e continuava articulando tranquilamente
a campanha governamental de produção e disseminação em massa de
desinformação para a manutenção do apoio ao seu pai. O número \textsc{ii} é
considerado membro da chamada ala olavista do bolsonarismo, composta
pelos seguidores de Olavo de Carvalho, considerado o guru
ultraconservador da família. Vale lembrar que Carlos Bolsonaro foi o
filho que desfilou na posse no carro do pai, afirmando que estava ali
para proteger o presidente de um possível atentado, mostrando sua
aproximação com as teorias conspiratórias.

Em diversas situações essa ação coordenada pode ser verificada e nos
momentos em que o governo passou por questionamentos em relação a algum
tipo de deliberação ou ato governamental, intensificava-se e se
extremava ainda mais a campanha desinformativa. Não havia nenhum tipo de
recuo estratégico, e mesmo quando parecia que o governo abandonaria alguma
postura ou decisão, se radicalizava a ação discursiva com base em
informações falsas e distorcidas.

O discurso antissistêmico tem sido um carro-chefe da desinformação
governamental. Não é novidade, desde a campanha eleitoral e até mesmo
antes dela, que Bolsonaro falava contra as instituições, mesmo com
sucessivos mandatos parlamentares. Foi eleito vereador em 1988 e teve
assento na Câmara dos Deputados entre 1991 e 2018. Sempre fez crítica às
instituições como se nunca tivesse participado delas, e além de sua
participação direita, sempre estimulou seus filhos a seguirem o mesmo
caminho e buscarem mandatos parlamentares.

O filho número \textsc{i}, Flávio Bolsonaro, é senador eleito em 2018, mas era
deputado estadual na Assembleia Legislativa do Rio de Janeiro.
Eduardo Bolsonaro, o número \textsc{iii}, é também eleito deputado federal em 2018,
após cumprir mandato de deputado estadual pela Assembleia Legislativa de São Paulo. 
O número \textsc{iv}, filho do segundo casamento, ainda não
tem nenhum mandato, mas tem demonstrado grande identificação com a forma
de atuar do pai e dos irmãos, principalmente no tráfico de influência
para benefícios próprios, de familiares e amigos. Os \textit{Bolsonaros} também
possuem relações íntimas com as milícias no Rio de Janeiro e há fortes
indícios de envolvimento no assassinato da ex-vereadora Marielle Franco
(\textsc{psol}) e o motorista Anderson Gomes pelas ligações com os denunciados
Ronnie Lessa, policial militar reformado, que antes de ser preso morava em
uma casa no mesmo condomínio do presidente, aparece em várias imagens ao
lado do mandatário Elcio Vieira de Queiroz.

Outro indício da lógica antissistêmica na retórica de Bolsonaro está na
crítica constante aos partidos políticos que ocupam um papel importante
na sociedade democrática, mesmo que cada vez mais enfraquecidos.
Bolsonaro se mostra despreocupado com os vícios da dinâmica partidária e
mais interessado em criar um ambiente em que se acentue a deslegitimação
das instituições que fazem parte do jogo democrático. Independente da
necessidade de uma reforma política que privilegie a existência de
partidos que não sejam legendas esvaziadas, mas que cumpram seu papel
como instância de participação dos cidadãos por se tratar de um
agrupamento que congrega opiniões de setores da sociedade.

A família Bolsonaro tentou, mas não conseguiu criar um partido, conforme
vemos na fracassada tentativa de registrar a Aliança pelo Brasil,
partido que poderia ser o primeiro da extrema-direita do país após a
redemocratização de 1985. Não é novidade que essa família trata as
agremiações partidárias como meras legendas de aluguel, mudando com
certa constância, sem nenhum compromisso político. As letras da legenda
eram compostas por letras escritas com balas de armas de fogo, símbolo
de um dos pilares do bolsonarismo que defende o armamento da população.
Haja vista os decretos presidenciais que flexibilizam a compra e o uso de
armas para qualquer cidadão e as constantes aparições ostentando armas
potentes.

A família Bolsonaro tem relação direta com o gabinete do ódio, grupo de
assessores que trabalha com as redes sociais no Palácio do Planalto,
fazendo gestão de diversas páginas que disseminam desinformação, boatos
e mentiras para atacar os adversários políticos do presidente, uma entre
outras atividades não institucionais que usam a estrutura pública para
coordenar ações que têm como foco desestabilizar a democracia.

A postura agressiva em relação às instituições e à ordem democrática era
meticulosamente calculada e, mesmo no momento em que a sociedade se
mostra mais fragilizada em função da realidade imposta pela pandemia do
\textsc{covid-19}, Bolsonaro participou de atos públicos com a pauta
antidemocrática em 2020. Foram manifestações organizadas em várias
cidades do país que reivindicavam o fechamento do Congresso Nacional e
do \textsc{stf}, defendia a intervenção militar e a
reedição do Ato Institucional n.\,5 (\textsc{ai-5}), decreto emitido pela Ditadura
Militar em 1968, responsável pelo período mais sombrio do regime com
medidas de endurecimento do combate aos seus opositores. Era a política
do medo, do cerceamento à liberdade de expressão.

Bolsonaro, além de participar das manifestações em frente ao Palácio do
Planalto, discursou e defendeu a legitimidade das reivindicações dos
manifestantes. O \textsc{stf} abriu inquérito para investigar a organização
desses atos e o processo revelou algumas pistas importantes para mostrar
a atuação de um assessor especial da Presidência, apontado como
integrante do gabinete do ódio. As investigações revelaram que o acesso a um
perfil do presidente no Instagram havia sido feito de equipamentos
conectados à Internet do Palácio do Planalto, o que comprova o uso da
máquina pública para promover a desinformação.
%e favorecer.

\section{Reviravoltas no cenário político de 2021}

O ano 2021 iniciou com a promessa de ser quente na política brasileira
após a cena dos primeiros vacinados no mundo e a da pressão de prefeitos e
governadores para a comercialização de doses de imunizantes. A vacina
não era mais ficção, tornava-se fato para a população que manifestava
querer o imunizante. Mas a necropolítica do governo Bolsonaro seguia com
a prática negacionista, promovendo aglomeração com seu populismo de
direita, não usando máscaras e vendendo recomendação ao uso da
cloroquina, que já se sabia ser ineficaz para tratar a \textsc{covid-19}. Com
muitos sinais trocados, aos poucos ficava explícita a política para
deixar a população morrer, como se não tivesse nenhuma responsabilidade
frente aos fatos.

O país vivia o colapso na saúde com crises em várias regiões do país por
falta de insumos básicos para os cuidados dos infectados impedindo o
atendimento mínimo das unidades hospitalares. Tudo isso somado às
elevadas taxas de desemprego, os altos custos da cesta básica e piora da
situação econômica do país, enquanto o presidente gozava de férias que
custaram aproximadamente 2,4 milhões de reais aos cofres públicos. A falta de
empatia com a situação era reiterada e, sempre que questionado sobre
as medidas a serem adotadas para minimizar o sofrimento da população,
respondia que não era seu problema, que nada tinha a fazer. Na verdade,
tinha muito o que poderia ser feito e mais que a falta de vontade
política para tomar as decisões e implantar as medidas que cabiam ao seu
cargo, como a compra de vacinas, mas preferia seguir com sua retórica de
responsabilizar adversários pela situação e fazer propaganda de
medicamento ineficaz para os cuidados da \textsc{covid-19}.

Para aumentar o cenário de adversidade para Bolsnoraro, Lula recuperou
os direitos políticos e se tornou elegível, após surpreendente decisão
do \textsc{stf}. O ex-presidente tentava provar sua inocência desde 2017 com os
mesmos argumentos de parcialidade do então juiz Sergio Moro, além de ter tido
sua reputação e imagem política atacada tanto pela condenação
constantemente realizada pelos meios de comunicação, um julgamento
midiático sem direito ao contraditório, princípio básico do jornalismo.

A reviravolta da condenação de Lula teve início em 2021, após as
denúncias reveladas pela agência de notícias Intercept Brasil que
mostrou através da troca de mensagens telefônicas entre Moro e o
procurador Deltan Dallagnol e a articulação para incriminar o
ex-presidente e impedi-lo de ser candidato nas eleições presidenciais de
2018. E é sempre bom frisar que foi um dos primeiros a ser convidado
pelo presidente para o Ministério da Justiça, convite rapidamente
aceito.

Atuando como uma espécie de tribunal midiático, a grande mídia atuava de
forma hegemônica, uma espécie de consenso fabricado, conforme reflexão
de Noam Chomsky, para tornar Lula o inimigo público número um do povo
brasileiro, por vezes fazendo coro com a retórica bolsonarista que
produzia diariamente desinformação, discurso de ódio às minorias e ódio
de classe. Um ambiente tóxico como jamais visto na política nacional. As
notícias falsas atuavam como alimento às narrativas mais reacionárias e
autoritárias.

O envolvimento de integrantes do governo, de seus filhos, sua mulher,
amigos e aliados históricos ao longo de dois anos e meio de mandato foi
fechando o cerco para Bolsonaro, e não paravam de aparecer fatos que
traziam à tona seu projeto político. A criação de uma Comissão
Parlamentar de Inquérito para investigar as ações e omissões do governo
federal no enfrentamento da pandemia da \textsc{covid-19} e o colapso dos
serviços públicos de saúde revelava que para além do negacionismo que
tentou minimizar a pandemia, atuar contra as medidas indicadas pelas
autoridades internacionais sanitárias, colocar em dúvida a eficácia e os
efeitos dos imunizantes, o governo não negociou a compra de vacinas com
alguns laboratórios, alegando não concordar com as regras estabelecidas,
mas fazia tratativas privilegiando outras empresas que vendiam a vacina
com preço mais elevado e sem que houvesse as devidas autorizações por
parte das autoridades competentes.

Mas não era só o negacionismo que fazia com que Bolsonaro inventasse e
espalhasse informações falsas sobre as vacinas, conforme investigações
em curso na \textsc{cpi} da \textsc{covid-19}. Começava a aparecer fortes indícios de
um esquema de corrupção dentro do Ministério da Saúde, envolvendo aliados
muito próximos do presidente que negociavam as vacinas a preços
superfaturados. Portanto, ao mesmo tempo que ``vendia dificuldades''
para negociar com alguns laboratórios usando a retórica que não
compraria a vacina a qualquer preço e não aceitaria as exigências
impostas por algumas empresas, combinava com outros laboratórios a
aquisição por valor ainda maior. Enquanto essa negociata ocorria, o
Brasil aparecia no ranking internacional entre os países que mais
registravam mortes pela doença.

\chapter[Fake news e a ordem do discurso desinformativo]{Fake news e a ordem do discurso\\desinformativo}

%\section{Algumas noções e conceitos}

Certa vez, Umberto Eco, escritor e crítico literário italiano, afirmou
em uma entrevista: todo fundamentalismo quase sempre se baseia em
afirmações falsas.\footnote{Ilze Scamparini. Todo fundamentalismo quase sempre se baseia em afirmações falsas. \textit{Consultório Jurídico}, 17 de julho de 2015.} Com esse alerta, Eco nos provoca a
pensar sobre os propósitos da produção e do compartilhamento da
desinformação e quais são os interessados nesse tipo de dinâmica (des)\,informativa.

As notícias falsas não são invenções recentes. Existem há muito tempo e
produzem desdobramentos que podem ser desde os mais inofensivos até
mesmo os mais fatais. O termo \textit{fake news} passou a ser utilizado de
forma quase banal, desqualificada e descontextualizada.

Nesse capítulo abordaremos algumas definições sobre a desinformação na
contemporaneidade, buscando analisar suas principais características,
seus diversos formatos e o papel das mídias digitais na sua
disseminação. Reconhecemos que as diferentes caracterizações sobre o
fenômeno não são consensuais, embora existam pontos de convergência e de
complementariedade entre elas. Sendo assim, não pretendemos esgotar o
debate.

Enfatizaremos a abordagem dos efeitos políticos da desinformação por
considerarmos que ele impacta nas diversas dimensões da vida. Isso
significa dizer que ao falarmos de questões relativas à saúde, por
exemplo da desinformação na pandemia da \textsc{covid-19}, verificamos claramente
um conjunto de estratégias e atores políticos disputando narrativas em
torno desse fenômeno e não apenas pela necessidade de se debater
políticas públicas para enfrentá-lo. Interesses diretos fizeram com que
certos discursos negacionistas ganhassem visibilidade em detrimento de
outros e exercessem influência importante no entendimento das populações
sobre os impactos do coronavírus.

Posetti e Matthews\footnote{Una breve guía de la historia
de las `noticias falsas' y la desinformación. \textit{\textsc{icfj}}, julho de 2018.} elaboram um levantamento sobre casos de
informações falsas e desinformação, demonstrando que o fenômeno não é
recente, mas atravessa diferentes períodos históricos. Embora saibamos
que o fenômeno sempre existiu, ainda assim é relevante distinguir suas
características ao longo da história, como, por exemplo, as diferentes
formas de produção, disseminação e os efeitos das notícias falsas. 
Consideramos que o alcance e os possíveis impactos podem chegar a ser 
extremamente danosos quando parcela significativa da sociedade está 
formando suas interpretações dos acontecimentos com base em falsificações.

Allcott e Gentzkow definem as \textit{fake news} como ``artigos
noticiosos que são intencionalmente falsos e aptos a serem verificados
como tal, e que podem enganar os leitores'',\footnote{Social media and fake news in the
2016 election. \textit{Journal of Economic Perspectives}, v.\,31, n.\,2, p.\,214, 2017.} portanto a falsificação é deliberada. A iniciativa de difundir uma
notícia falsa pode partir de adversários que pretendem destruir a
reputação de um concorrente para se apresentar como mais vantajoso,
fenômeno que pode acontecer na política, em relações comerciais e até
mesmo em círculos sociais nos quais se pretenda conseguir alguma
vantagem.

Wardle e Derakhshan\footnote{\textit{Information
Disorder: Toward an interdisciplinary framework for research and policy
making}. Estrasburgo: Council of Europe, 2017.} também descartaram o uso de \textit{fake news}
por considerarem que a apropriação feita pelos políticos causa confusão
deliberada, muitas vezes para esconderem a própria rede de informações e
de notícias falsas. Para dar conta da complexidade informacional, as
autoras elaboraram a seguinte tipologia:

\begin{itemize}
\item \textit{Dis-information}: informações falsas criadas deliberadamente para
prejudicar uma pessoa ou instituições;

\item \textit{Mis-information}: informações falsas, mas que não foram criadas
com a intenção de danificar a imagem de pessoas ou de instituições;

\item \textit{Mal-information}: são as informações corretas, mas usadas fora de
contexto para causar algum tipo de dano.
\end{itemize}

O Conselho Europeu, órgão da União Europeia para definir orientações e
prioridades políticas, desempenha um papel fundamental nesse debate e
adotou a seguinte definição para o fenômeno das notícias falsas:
\textit{mis-information} ou informações enganosas são aquelas
compartilhadas sem a intenção de causar algum tipo de dano a alguém;
desinformação ou \textit{disinformation} é quando se sabe que a informação
é falsa e o compartilhamento ocorre com algum objetivo nos possíveis
efeitos; e, por fim, a má-informação ou \textit{mal-information} que são
informações geradas em âmbito privado, mas com caráter verdadeiro e
quando compartilhadas publicamente podem causar dano individual ou
coletivo.

A Comissão Europeia criou o Grupo de Alto Nível sobre Notícias Falsas e
Desinformação Online.\footnote{Em inglês, \textit{High Level Group on Fake News and Online
Disinformation}.} O grupo é seguido por outras instituições em âmbito
internacional, entre elas no Brasil, o Comitê Gestor da
Internet,\footnote{O Comitê Gestor da Internet foi criado pelo decreto
  número 4829 de 3 de julho de 2003, com a atribuição de estabelecer as diretrizes
  relacionadas ao uso do desenvolvimento da Internet no Brasil com as
  diretrizes para a execução do registro de Nomes de Domínio, alocação
  de Endereço de \textsc{ip} (Internet Protocol) e a administração pertinente ao
  Domínio de Primeiro Nível \textit{br}.} que vem desempenhando um papel
importante nas discussões da regulação da desinformação. O Grupo de Alto
Nível parte do princípio de que todas as formas de informação falsa,
imprecisa ou enganosa apresentam alguma intencionalidade. Preocupados
também com as constantes ameaças à liberdade de expressão, os autores do
referido relatório afirmaram a necessidade de garantirmos no debate a
proposição de medidas capazes de proteger a diversidade e a
sustentabilidade do ecossistema midiático, de ampliar e promover
iniciativas de educação midiática, além de garantir o incentivo à
pesquisa continuada sobre os impactos da desinformação.

Os itens acima destacados dizem respeito à preocupação com possíveis
excessos nas propostas regulatórias para enfrentar a desinformação,
preocupação presente no cenário brasileiro quando diversos projetos de
lei foram apresentados sobre o tema. Nesse sentido, é afirmada a
necessidade de colaboração multissetorial para minimizar o
intervencionismo regulatório de governos, com base em princípios
definidos de forma transparente. Portanto, as definições de tais medidas
não devem ficar circunscritas no âmbito do parlamento e de outras
estruturas do Estado, mas devem envolver os mais diversos setores da
sociedade civil de forma transparente e participativa.

\section{Plataformas digitais e a circulação da desinformação}

As plataformas desempenham um papel significativo na contemporaneidade
considerando que através delas circulam os fluxos informacionais com os
mais variados conteúdos e formatos. A expansão de seus modelos de
negócios deve ser contextualizada a partir da dinâmica do capitalismo
neoliberal, cujo estabelecimento da lógica de mercado influencia as
ações algorítmicas. À medida que os conteúdos são analisados por sua
capacidade de propagação, tendem a predominar aqueles que conseguem 
atingir grande repercussão, ou grande visibilidade nas redes digitais.

% ñ está na bibliografia:  (Newman, Fletcher, Kalogeropoulos, Levy \& Nielsen, 2019)
A relação entre as plataformas e a disseminação de desinformação também
é destacada no relatório Digital News Report, do Reuters Institute. Em 2019, a
equipe de pesquisadores realizou um estudo que apresentou informações
importantes para a compreensão dos impactos do fenômeno da
desinformação, particularmente no Brasil. Os dados apontaram que a
comunicação social de notícias está cada vez mais no âmbito privado e
foi confirmado o crescimento dos aplicativos de mensagens. Nesse
sentido, o estudo demonstrou o crescimento explosivo do WhatsApp na
América Latina, particularmente no Brasil. Aproximadamente 53\% da
amostra de usuários brasileiros usavam essa plataforma para se informar
em comparação aos 9\% no Reino Unido, 6\% na Austrália e 4\% nos \textsc{eua}.

Entre as principais preocupações apontadas pelo relatório está a
capacidade de se distinguir uma notícia com base factual e a
desinformação e, mais uma vez, o Brasil liderou o ranking dos países em
que esse discernimento esteve quase ausente. Certamente, a dificuldade
em diferenciar informações com bases factuais ou não é bastante
preocupante quando pensamos que essa realidade deixa os indivíduos
vulneráveis aos conteúdos com a intencionalidade de distorcer a
realidade.

A pesquisa revelou o crescimento dos aplicativos de mensagens privadas
como, por exemplo, o WhatsApp, que passou a rivalizar inclusive com o
Facebook no compartilhamento de notícias, embora ambas sejam de
propriedade de Mark Zuckerberg. Uma das diferenças do ponto de vista da
circulação de desinformação é que o Facebook é uma mídia pública e o
WhatsApp, fechada. Isso significa que é difícil mensurar o número de
grupos fechados existentes. O aplicativo foi apontado como um dos
maiores propagadores de desinformação e alguns pesquisadores
desenvolveram metodologias para analisar o fenômeno, embora haja
dificuldade para se pesquisar esses dados.

As formas de disseminação das notícias foram modificadas com a presença
cada vez maior das tecnologias informacionais. O uso de robôs,
algoritmos, inteligência artificial incidem diretamente nessa realidade:

\begin{quote}
As notícias falsas podem ser consideradas não apenas em termos da forma
ou conteúdo da mensagem, mas também em termos de infraestruturas
mediadoras, plataformas e culturas participativas que facilitam a sua
circulação. Nesse sentido, o significado das notícias falsas não pode
ser totalmente compreendido fora da sua circulação \textit{online}.\footnote{Liliana Bounegru \textit{et al.} \textit{A Field Guide to Fake News: a collection of recipes for those who love to cook with digital methods}. Amsterdã: Public Data Lab, 2017.}
\end{quote}

Tecnicamente, um algoritmo é uma sequência de regras ou instruções
voltadas para a execução automatizada de uma tarefa. As operações
algorítmicas nas plataformas digitais exercem funções complexas,
utilizadas para as mais variadas finalidades, de forma pouco
transparente e cada vez mais autônoma. A partir da correlação de
variados tipos e fontes de dados, os algoritmos operam diferentes formas
de classificação, segmentação, visualização, processamento de
informação, recomendação, reconhecimento de padrões individuais e
relacionais, sendo responsáveis tanto por extrair o valor dos dados
quanto por toda a oferta de um mundo visível, personalizado de ações e
interações para os usuários. Nesse sentido, os modelos de
previsibilidade e os sistemas de recomendações dos algoritmos exercem um
papel central na lógica da indústria da influência, uma vez que são eles
que operacionalizam a promessa de previsão dos comportamentos futuros e
direcionam formas de intervenção em tempo real sobre tais condutas.

Ao contextualizarmos o debate sobre a desinformação na chamada era
informacional, nos preocupamos em pensar principalmente nos impactos no
campo político. Morozov nos chama a atenção para um aspecto perturbador
quando diz que ``a democracia está se afundando em \textit{fake news}'' e ainda indaga ``será que a crise das \textit{fake news} é
a causa do colapso da democracia? Ou seria ela só a consequência de um
mal-estar mais profundo, estrutural, que está em desenvolvimento há
muito tempo?''.\footnote{Evgeny Morozov. \textit{Big Tech: a ascensão dos dados e a
morte da política}. São Paulo: Ubu, 2018, p.\,182-183.} Dessa forma, na era
informacional os algoritmos assumem um papel cada vez mais de destaque e
transformam os aspectos mais banais de nossa vida cotidiana em ativos
rentáveis, monetizáveis, que o autor chama de capitalismo dadocêntrico.

As plataformas de redes sociais dependem dos dados dos indivíduos para
alimentar sua cadeia produtiva para a geração de bilhões de lucro. Todas 
nossas ações na Internet são armazenadas, processadas e analisadas, e
podem ser utilizadas em diversas situações, sem que saibamos que isso
está acontecendo. Com essas informações frequentemente disponibilizadas
é possível a criação de perfis de comportamentos para compreender os
interesses, gostos, ou seja, é possível predizer as atitudes dos
indivíduos e aplicar as técnicas de psicometria para a identificação das
personalidades.

É por isso que nossos dados são o bem mais valioso da contemporaneidade
e devemos nos empenhar ao máximo para a criação de mecanismos para
evitá-los, garantindo a privacidade dos indivíduos. Silveira
afirma que:

\begin{quote}
A sociedade informacional, ao expandir as tecnologias de armazenamento,
processamento e distribuição de dados, ao gerar uma intensa
digitalização de nossos registros cotidianos, tornou o mercado de dados
um dos segmentos mais importantes da economia mundial e revigorou as
promessas positivistas abaladas pela crise dos paradigmas da ciência. O
fluxo intenso de dados e um capitalismo altamente concentrador de
riqueza orientaram o desenvolvimento tecnológico na direção da
personalização das vendas. A busca de compradores é, antes de mais nada,
a procura de dados sobre cada um deles. Esses dados sobre o
comportamento, o gosto e os detalhes do passado e do presente geram as
informações necessárias para a captura desses consumidores e o
conhecimento do que os agrada e os encantará no futuro.\footnote{Sergio Amadeu da Silveira. \textit{Democracia e os códigos
invisíveis: como os algoritmos estão modulando comportamentos e escolhas
política}. São Paulo: Edições Sesc, 2019.}
\end{quote}

%fBig mesmo? Não encontrei na bibliografia Zuboff  (2018)
Esse mercado de dados tem sido alimentado pelos sistemas de \textit{Big Data}
dominados pelo Google, Facebook e Amazon, para falar dos mais
expressivos. Em estudo aprofundado sobre o Google, Shoshana Zuboff, da
Harvard Business School, mostra como a plataforma foi a pioneira
no uso do \textit{Big Data} e, consequentemente, a protagonista na lógica de
acumulação, definida como capitalismo de vigilância. Para a autora, o
\textit{Big Data} é, ao mesmo tempo, condição para o processo de
acumulação quanto sua própria expressão ao implantar uma lógica de
acumulação que substitui os contratos, o Estado de direito pela
soberania do Big Other que atua justamente na ausência de uma autoridade
legítima e acaba exercendo um poder imenso em relação aos cidadãos.
Nesse sentido, o Big Other seria uma espécie de atualização do Big
Brother, do romance 1984, de George Orwell.

\section{Semeando a desconfiança}

% ñ encontrei na bibliografia Guess, Nyhan e Reifler (2018)
Guess, Nyhan e Reifler falam de ``um novo tipo de desinformação
política'' marcada por uma ``dubiedade factual com finalidade
lucrativa''. Os autores destacam cinco questões centrais para a
desinformação na política. A primeira é que as pessoas consomem notícias
que reforçam suas opiniões e seus pontos de vista sobre diferentes
aspectos da realidade. Essa dinâmica faz implodir pontes, diálogos,
debates com alguma consistência para além de meras opiniões infundadas.
Esse aspecto pode ser perigoso dependendo da amplitude de sua expansão.
Pode chegar a linchamentos de reputação e até mesmo de eliminação física
daquele de quem se discorda, gerando um ambiente hostil e polarizado.

% Não encontrei na bibliografia (Baldacci, Buono e Grass, 2017)
Também vemos que, em muitos casos, as pessoas confiam em opiniões de
pessoas ou grupos socialmente influentes, que teriam legitimidade para
manifestarem o que pensam. O aspecto
emocional no compartilhamento de informações duvidosas ou falsas é muito
importante por criarem a sensação nos indivíduos de fazerem parte do
processo que está sendo discutido, de demonstrarem que são
bem-informados. Trata-se de comportamento frequentemente observado nas
redes. Muitas vezes a informação é transmitida para outras pessoas sem
que seja lida ou se saiba claramente se há alguma base concreta no que
está sendo dito.

Observar a configuração da dieta informacional é muito importante para
compreendermos os motivos pelos quais estamos expostos a um conjunto
específico de informações que circulam nas redes digitais e não a
outros, caracterizando uma forma de direcionamento. Trata-se do filtro
bolha definido a partir das ações algorítmicas que personalizam o
conteúdo que acessamos, por meio dos dados que disponibilizamos em nossa
navegação nas mais diversas plataformas. Para essa atividade se utiliza
o chamado viés de confirmação, ou seja, uma tendência cognitiva mostra
que as pessoas costumam ter uma propensão a prestar atenção naquilo que
confirma suas crenças e, frequentemente, ignorar o que contradiz sua
visão de mundo.

Parisier, ao estudar o tema identificou que a bolha de filtros
traz três novas dinâmicas, tais como:

\begin{quote}
Cada pessoa está sozinha em sua bolha. Numa época em que as informações
partilhadas são a base para a experiência partilhada, a bolha dos
filtros é uma força centrífuga que nos afasta uns dos outros. Segundo, a
bolha dos filtros é invisível. Os espectadores de fontes de notícias
conservadoras ou progressistas geralmente sabem que estão assistindo a
um canal com determinada inclinação política {[}\ldots{]} Por fim, nós não
optamos por entrar na bolha {[}\ldots{]} não fazemos esse tipo de escolha quando
usamos filtros personalizados. Eles vêm até nós --- e, por serem a base
dos lucros dos sites que os utilizam, será cada vez mais difícil
evitá-los.\footnote{Eli Parisier. \textit{O filtro invisível: o que a internet está
escondendo de você}. Rio de Janeiro: Zahar, 2012, p.\,11-12.}
\end{quote}

Para o autor, a influência do que vemos na Internet parte das
preferências registradas em nossa navegação nas diversas plataformas.
Esse tipo de filtragem leva ao desaparecimento de visões de mundo
opostas e no contexto político leva ao bloqueio do debate entre
concepções diferentes. A ausência de contato com opiniões divergentes
faz com que os indivíduos fiquem mais suscetíveis a preconceitos,
opiniões extremadas e, principalmente, à desinformação, além de corroer
um aspecto fundamental da sociedade democrática que é o debate de
ideias.

Assim como a desinformação, a produção de rumores, boatos e fofocas não
são fenômenos recentes e estão em várias dimensões da vida social e
política. O sociólogo Norbert Elias abordou o papel da fofoca em uma de
suas célebres obras, \textit{Outsiders e estabelecidos}, na qual estudou
as relações de uma comunidade no interior da Inglaterra. Para o autor, a
fofoca sempre tem dois polos, aqueles que a circulam e aqueles que
seriam os alvos delas, e se vincula com a estrutura e as relações dos
grupos existentes na comunidade em que se circula. No caso da comunidade
inglesa é importante destacarmos que a etnografia realizada por Elias
estava relacionada a uma série de disputas existentes na comunidade e a
circulação de informações era realizada no chamado boca a boca, sem
nenhuma forma de mediação.

Se transportarmos a análise para as mídias digitais verificaremos em
alguma medida a presença desse tipo de comportamento nos usuários. O uso
de \textit{blame gossip}\footnote{A expressão diz respeito à fofoca depreciativa, e \textit{praise gossip}, à elogiosa.} também pode ser verificada no compartilhamento de
desinformação nas redes para atingir as reputações, mas também funciona
como forma de manter os laços sociais entre os integrantes de uma
comunidade.

\section{Problematizando a pós-verdade}

A noção de pós-verdade também passou a fazer parte dos debates sociais e
políticos e chegou a ganhar um verbete no dicionário Oxford que a
definiu como um adjetivo relacionado às circunstâncias nas quais os
fatos objetivos são menos influentes na formação da opinião pública do
que os apelos às emoções ou crenças pessoais. As razões alegadas para a
inclusão do termo remetem ao contexto da eleição presidencial nos
Estados Unidos, bem como ao referendo do Brexit que aprovou a
saída do Reino Unido da União Europeia. Nesse sentido, alguns
pesquisadores começaram a se questionar se seria, então, o campo
político responsável pela disseminação da ideia de que estamos na era da
pós-verdade.\footnote{Arthur Bezerra, Rafael Capurro e Marco Schneider. Regimes de
verdade e poder: dos tempos modernos à era digital. \textit{Liincem Revista}, Rio de Janeiro, v.\,13, n.\,2, p.\,371--380, novembro de 2017.}

A perspectiva de Michel Foucault talvez seja uma mais potente para
problematizar a expressão pós-verdade. O autor não se propõe a definir o
que é verdade, mas traz à tona a importância do regime de verdades de
cada sociedade. A partir disso fica claro que em cada dinâmica social,
em cada momento histórico, se acolhe um tipo de discurso como
verdadeiro. Significa dizer que o discurso adotado como verídico não
está isento de interesses políticos, econômicos ou científicos. Para o
autor, o discurso é o próprio campo de disputa, de enfrentamentos
sociais, embates políticos, onde os pontos de vista estão em discussão.

O debate em torno da pós-verdade vem ganhando espaço importante em
circuitos do noticiário e entre intelectuais de diferentes perspectivas
analíticas. Frequentemente nesses debates verificamos uma associação
entre os filósofos Nietzsche, Foucault e Derrida como os precursores da
noção de pós-verdade, colocando-os em um mesmo campo de ideias que os
produtores de desinformação e propagadores de mentiras.

A esse tipo de falsificação é importante deixar claro que não
são da perspectiva negacionista e por isso é importante qualificar a
contribuição desses filósofos para a compreensão da relação entre moral
e ciência. Para Nietzche, caberia à ciência dissipar os erros e as
ilusões da razão sempre que se apresentasse contaminada pela moral
religiosa, portanto, preocupava-se com a subordinação do conhecimento à
moral. É a isso que o filósofo convocava os homens de \textit{espíritos
livres} a se manterem na busca pela ampliação do conhecimento
científico, livres da moral e abertos às experimentações.

A pós-modernidade tem papel fundamental na problematização da ciência e
é preciso separar a crítica à ciência, desenvolvida por
pós-estruturalistas e pós-modernistas, das práticas e discursos da
pós-verdade realizada pela extrema-direita que se utiliza de um discurso
anticientificista, anti-intelectualista e negacionista para justificar
seu posicionamento que se aproxima de estratégias e práticas de
intolerância, tais como o ódio, o sexismo, o racismo, entre outras.
Qualquer tipo de associação da noção de pós-verdade com a reflexão
desses autores deve ser evitada para que não se criminalize uma forma de
pensamento tão potente que contribui significativamente para a reflexão
da contemporaneidade.

Nietzsche antecipou críticas aos modelos científicos
predominantes durante parte significativa dos séculos \textsc{xix} e \textsc{xx}, e por
isso é considerado um pensador intempestivo, que esteve fora de seu
tempo, antecipando problematizações que passaram a fazer parte do debate
científico contemporâneo. O conhecimento científico é aspecto básico
para a realização da pesquisa como instrumento de investigação dos mais
diversos fenômenos.

Nesse sentido, a crítica, o questionador espírito científico
possibilita a condução ao novo. A problematização fundamentada em um
conjunto de pressupostos e de evidências faz parte do debate do campo
científico e essa dinâmica não tem nenhuma relação com o negacionismo
científico observado no período histórico recente, como por exemplo, a
indagação desenvolvida pelos terraplanistas que negam evidências
apresentadas sobre a esfericidade da Terra, um dos campos de estudo mais
antigos, inaugurado pelos gregos aproximadamente em 300 a.\,C.

\section{Iniciativas de enfrentamento à desinformação}

A expansão da produção do compartilhamento de notícias falsas,
desinformação, discursos de ódio e intolerância trouxe um processo
de transformação social que precisa ser refletido e debatido com a
participação dos mais diversos setores da sociedade preocupados com os
efeitos e impactos desse fenômeno na ordem democrática. A capacidade de
segmentação dos usuários está provocando impacto significativo na
dinâmica de configuração da opinião pública, cada vez mais modulada e
modelada a partir dos dispositivos informacionais com base em
\textit{machine learning} orientada em interesses econômicos e
políticos.

Embora seja preocupante o crescimento da produção e ampla disseminação
da desinformação, é muito importante também destacar as iniciativas de
enfrentamento a esse fenômeno. Abordaremos algumas das iniciativas de
diversos segmentos da sociedade civil, de pesquisadores, dos
parlamentos, do campo comunicacional, no sentido de debater e criar
mecanismos para alertar a sociedade sobre os perigos desse processo,
apresentando algumas possibilidades de enfrentá-los.

\subsection{\textsc{cpmi} das \textit{fake news} e propostas legais}

Instalada em 4 de setembro de 2019, a Comissão Parlamentar Mista de
Inquérito (\textsc{cpmi}) das \textit{fake news} tinha como objetivo apurar elementos
ligados às notícias falsas durante as eleições de 2018 no Brasil. Entre
as atribuições estava a investigação sobre a criação de perfis falsos
para influenciar as eleições, os ataques cibernéticos realizados no
período e a prática de \textit{ciberbullying} contra autoridades e
cidadãos. A comissão tinha natureza mista, portanto, contava com a
participação de deputados federais e senadores, sendo 16 titulares e 16
suplentes.

Durante a fase de proposição foram coletadas informações através de
solicitação e recebimento de documentos com os seguintes conteúdos:
denúncias e investigações; dados bancários; dados telefônicos; estudos e
pareceres técnicos de pesquisadores; pareceres técnicos oriundos de
plataformas presentes na Internet, tais como Facebook, Instagram e
YouTube, de sites e mesmo do WhatsApp. Outra fonte de levantamento de
informações durante a fase de proposição foram as audiências públicas e
oitivas realizadas com depoentes diretamente relacionados ao evento
tratado, ou mesmo com especialistas vindos da comunidade científica, do
setor empresarial e da sociedade civil organizada.

Desde o início das atividades a \textsc{cpmi} convocou pessoas para prestarem
depoimentos, alguns considerados fundamentais para o esclarecimento de
denúncias sobre o uso de \textit{fakes news} e discurso de ódio, além do
levantamento de documentos fundamentais para as investigações em
andamento.

Entre os depoimentos, destacamos aqueles de parlamentares eleitos na
coligação de Jair Bolsonaro, porém, que em poucos meses de governo
romperam com o presidente e se tornaram adversários políticos, embora
mantivessem os princípios ideológicos. São eles, Alexandre Frota,
atualmente filiado ao \textsc{psdb}, e Joyce Hasselmann, do \textsc{psl}. %embora sinalizasse possibilidade de mudança.
Ambos denunciaram as chamadas milícias digitais com ataques virtuais a 
adversários e oponentes do
governo e chamavam a atenção para a existência de um grupo instalado
dentro do Palácio do Planalto, conhecido como ``gabinete do ódio'',
composto por assessores especiais da Presidência para propagar notícias
falsas e campanhas difamatórias, sob a coordenação dos filhos do
presidente, Carlos Bolsonaro (Patriotas) e o deputado federal Eduardo
Bolsonaro (\textsc{psl}).

Embora com atividades praticamente paralisadas em virtude do isolamento
social, a \textsc{cpmi} realizou algumas atividades, com seus integrantes
participando de debates públicos, como por exemplo na discussão sobre a 
criação de legislação para o enfrentamento de notícias falsas e desinformação.

\subsection{Projetos de lei para enfrentar a desinformação}

Identificamos aproximadamente cinquenta projetos de lei apresentados na
Câmara dos Deputados com propostas para o enfrentamento das notícias
falsas e da desinformação. Elaboramos uma síntese contendo os principais
aspectos dessas proposições que expressam tanto a preocupação dos
parlamentares com os impactos sociopolíticos do fenômeno quanto a
necessidade de oferecerem algumas diretrizes sobre o tema para suas
bases. As proposituras foram agrupadas nos seguintes eixos:

\paragraph{Criminalização da prática de desinformação} O foco das proposituras versa sobre a criminalização da conduta de
produção e disseminação de notícias falsas e desinformação, e significa
tornar crime e atribuir pena aos indivíduos identificados nessas
práticas. Nesse aspecto também surgiram iniciativas para incriminar
especificamente a produção desinformativa relacionada à área da saúde.

\paragraph{Alterações em legislações} Diferentes projetos propõem alterações no código eleitoral com o objetivo de tipificar como crime a divulgação de fatos sabidamente inverídicos por candidatos em qualquer pleito. Foram identificados alguns projetos criminalizando a incitação da população pelas redes sociais que atentem contra a segurança nacional e à ordem pública e social.

Nesse item destacamos a proposta de regulamentação do Marco Civil da Internet, lei número 12.965/\,14 que estabelece princípios, garantias, direitos e deveres para o uso da Internet no Brasil. A proposta de regulamentação do governo Bolsonaro é considerada inconstitucional por especialistas, apresenta ambiguidades e pode criar a insegurança jurídica em empresas de vários setores.

Por outro lado, apesar do Marco Civil ser considerado uma das legislações mais avançadas do mundo é preciso atualizá-lo, tendo em vista a necessidade do estabelecimento de regulação para os aplicativos e redes sociais que na atualidade são os novos intermediários que filtram conteúdos por ações algorítmicas e inteligência artificial.

É necessário investir em medidas que aumentem a transparência das plataformas, cuja atuação é considerada bastante opaca nas ações adotadas em relação, por exemplo, à moderação de conteúdo. O decreto governamental é burocrático e mais confunde que contribui para tornar essas medidas mais claras.

\paragraph{Moderação de conteúdos} Nesse aspecto, identificamos a proposição de medidas para o enfrentamento de conteúdos de ódio e preconceito com o objetivo de diminuir esse tipo de prática nas redes. O debate em torno dessas questões é importante, embora sejam necessárias a ampliação e a diversificação de participantes da sociedade civil, principalmente, pelo fato de considerarmos que várias dessas medidas já estão tipificadas em lei, como por exemplo a
criminalização da prática de racismo.

Outra propositura polêmica está presente na sugestão de alteração do Marco Civil. O projeto de lei propõe condicionar a postagem de conteúdos nas redes sociais a partir do fornecimento prévio de número telefônico ou endereço de correio eletrônico. Essa é uma medida bastante rebatida pelos movimentos em defesa dos direitos de proteção de dados individuais, à medida que ele veda a prática do anonimato, portanto, a privacidade.

\paragraph{Educação digital} A educação digital também está presente em projetos de lei e um dos enfoques é a inclusão no currículo das escolas dos ensinos fundamental e médio de disciplina sobre utilização ética das redes sociais contra a divulgação de notícias falsas. Cabe lembrar que esse tipo de iniciativa vem sendo praticada em diversos países europeus, contribuindo para formar cidadãos comprometidos com a qualidade da informação.

\paragraph{Campanhas de esclarecimentos sobre notícias falsas} A instituição de campanhas de esclarecimento sobre notícias falsas e desinformação está presente em projetos que propõem formas de divulgação de temas como liberdade, responsabilidade e transparência na Internet, e também a criação de uma semana de conscientização sobre o enfrentamento à desinformação.

Consideramos essas iniciativas importantes para manter o debate presente sobre o tema, mas, evidentemente, devem ser associadas a outras propostas para que não se tornem projetos sem efetividade no debate social.

\paragraph{O projeto de lei das \textit{fake news}} O \textsc{pl} 2.630/\,20 de autoria do senador Alessandro Vieira (Cidadania), conhecido como ``\textsc{pl} das \textit{fake news}'', aprovado no Senado, ainda deve passar por discussão e possíveis alterações na Câmara dos Deputados. Trata-se de uma proposta polêmica que afeta os direitos dos usuários na Internet em vários aspectos.

Entre os pontos mais polêmicos, destacamos o potencial prejuízo aos direitos fundamentais como a privacidade, a proteção de dados, o acesso à Internet e a liberdade de expressão. Além disso, fere outras legislações com a proposta de criar a \textit{conta identificada}, quando o titular da mesma é plenamente identificado pelo provedor de aplicação a partir de confirmação de dados fornecidos previamente. Esse tipo de noção atinge frontalmente a privacidade e confere aos provedores um poder que não é desejável que possuam.

Outra questão polêmica é a \textit{identificação em massa} que prevê formas de indicar as chamadas contas inautênticas e deixa uma porta aberta para a massificação da coleta de informações pessoais.

O \textit{poder de polícia} das plataformas também é ponto controverso e está na contramão da Lei Geral de Proteção de Dados que pressupõe coleta mínima dos dados pessoais. Nessa perspectiva, a rastreabilidade em massa também é um risco caso a lei seja aprovada sem modificações, considerando que os aplicativos terão que guardar os dados das pessoas.

Há riscos também à liberdade de expressão pelas brechas que aumentam o poder das plataformas sobre o fluxo informacional e também contrariam as diretrizes estabelecidas na Lei Geral de Proteção de Dados.

\section{Discurso de ódio, desinformação e racismo}

Algumas medidas de contenção da desinformação nas redes sociais ganharam
força e estão promovendo um debate importante sobre a responsabilidade
em relação às condutas relacionadas à desinformação nas redes, quanto ao
financiamento das notícias falsas. Em diversos países começam a surgir
iniciativas nessa perspectiva, entre eles se destacam: Sleeping Giants,
Fakeing News: Fraudullent news and the fight for truth, Global Council
to Build Trust in Media and Fight Misinformation, Stop Hate for Profit, Global Desinformation Index (\textsc{gdi}), Jornalism Trust Iniciatitiva, entre
as outras.

Destacamos algumas dessas iniciativas para compreendermos a metodologia
e o objetivo do trabalho realizado:

\subsection{Índice global de desinformação (\textsc{gdi})}

É importante ressaltar que a desinformação pode gerar lucro e segundo o
\textsc{gdi} existe um mercado de
aproximadamente 235 milhões de dólares em publicidade \textit{online}, para ficarmos
com uma estimativa conservadora, conforme declaram os organizadores da
entidade. Nesse sentido, podemos afirmar que a desinformação tem sido
fonte geradora de grandes taxas de lucro para as plataformas e na
análise de Srnicek,\footnote{\textit{Platform Capitalism}. Cambridge: Polity Press, 2016.} proporcionando a ampliação do poder do capitalismo de plataforma.

O relatório divulgado pelo \textsc{gdi} demonstra como a chamada propaganda
programática ou mídia programática organiza os chamados leilões para
espaços publicitários nos sites que tornam disponíveis espaços para
anúncios e os anunciantes disputam esse espaço, considerando a
visibilidade que ele pode oferecer para a marca e seus produtos.

O Google domina esse mercado com aproximadamente 70\% dos sites de
desinformação com uma receita acumulada que gera em torno de \textsc{us}\$ 87
milhões. Outras empresas que também oferecem esse tipo de serviço são a
AppNexus e Criteo.\footnote{Global disinformation index. \textit{The Quarter Billion Dollar
Question: How is Disinformation Gaming Ad Tech?}. Reino Unido: setembro de 2019.}

\subsection{Sleeping giants: desmonetizando a desinformação}

Sleeping Giants foi criado nos Estados Unidos em novembro de 2016, sob o
impacto do uso de \textit{fake news} e desinformação durante as eleições
presidenciais. O objetivo era constranger as marcas que anunciavam em
páginas na Internet e influenciadores digitais considerados intolerantes
ou sexistas.

A estratégia se baseia em mostrar aos usuários um conjunto de
informações sobre as formas de financiamento dessas mídias. O objetivo
era questionar as empresas sobre o uso de desinformação ou notícias
falsas nos espaços onde estavam compartilhando suas marcas e seus
produtos. Isso fez com que essas plataformas tivessem prejuízos de
milhões de dólares com esse tipo de publicidade. Em poucos meses,
aproximadamente 4.500 anunciantes removeram seus anúncios do jornal
\textit{Breitbart News}, considerado o mais importante veículo da extrema-direita norte-americana, um dos maiores compartilhadores de mentiras,
notícias falsas, desinformação e discurso de ódio.

A publicidade programática é a substituição de atividades humanas na
negociação de espaços publicitários por diferentes tecnologias
automatizadas que ampliam as formas de se exibir um anúncio. O
publicitário Matt Rivitz observava atentamente essas estratégias
publicitárias e se preocupava com a destruição de reputação das marcas
que, em muitos casos, desconhecia onde seus logotipos estavam sendo
inseridos.

A partir dessa preocupação, resolveu criar a iniciativa Sleeping Giants
nos Estados Unidos. Até o presente momento, verifica-se que a iniciativa
passou a ser realizada nos seguintes países: Austrália, Bélgica,
Canadá, Finlândia, França, Alemanha, Itália, Holanda, Nova Zelândia,
Noruega, Espanha, Suécia, Suíça, Reino Unido e Brasil.

Esse modelo de negócios opera com os algoritmos de inteligência
artificial e com as operações que usam \textit{machine learning} ou
\textit{deep learning} possibilitando a potencialização e o gerenciamento
de dados obtidos das múltiplas informações dos usuários, principalmente
em atividades de consumo nas mídias digitais. Segundo publicitários
experientes, a escala de processamento de dados realizada pelas técnicas
de Inteligência Artificial é muito maior do que aquelas que os humanos têm
capacidade de realizar. Outra vantagem apontada é que esse tipo de
terceirização diminui o trabalho dos profissionais de marketing que
poderiam passar a se dedicar a pensar em outras estratégias e no
processo de criação, embora seja importante ressaltar que esse fator não
pode retirar a responsabilidade das empresas sobre os espaços em que
estão compartilhando seus anúncios.

Uma das principais vantagens destacadas pelos profissionais de
publicidade sobre esse modelo de negócios é que ele alcança o grau
máximo de precisão. Significa dizer que é um modelo capaz de fazer uma
predição para saber o meio em que o anúncio deve aparecer, qual o tipo
de anúncio e até mesmo a melhor hora para ampliar a audiência.

O modo de atuação se baseia na verificação dos anúncios alocados por
meio da ferramenta publicitária Google Adsense em sites de
desinformação. Os articuladores do Sleeping Giants realizavam alertas às
empresas, pois assim poderiam ficar cientes dos lugares em que seus
produtos estavam sendo publicizados. A partir daí ficava mais fácil
verificar se o produto ou marca estava associando sua imagem a um portal de
notícias que pratica desinformação. Outra atividade importante é a
criação de uma espécie de ``lista negra'' de empresas para impedir que a
propaganda tenha exposição.

Embora o Google tenha um papel protagonista nesse modelo de negócios
publicitários, o porta-voz declarou:

\begin{quote}
Existem políticas contra conteúdo enganoso em nossas plataformas e trabalhamos
para destacar conteúdos de fontes confiáveis {[}\ldots{]} Entendemos que os
anunciantes podem não desejar seus anúncios atrelados a determinados
conteúdos, mesmo quando não violam nossas políticas. Nossas plataformas
oferecem controles robustos que permitem o bloqueio de categorias de
assuntos e sites específicos, além de gerarem relatórios em tempo real
sobre onde os anúncios foram exibidos. 
%não encontrei na bibliografia: (MEIO \& MENSAGEM, 2020, \textit{online}).
\end{quote}

Evidentemente, é difícil para o maior buscador de informações do mundo
reconhecer sua responsabilidade sobre a publicidade em plataformas que
claramente são identificadas como produtoras e disseminadoras de
notícias falsas e desinformação, considerando que isso significa admitir que
as práticas empresariais estão associadas à geração de receita de
influenciadores ultraconservadores e de extrema-direita que afetam a
sociedade democrática.

Alguns filtros de controle podem ser adotados pelas empresas para evitar
que seus anúncios sejam veiculados para determinados grupos e esse
controle pode ser feito por meio de palavras-chave que são utilizadas
para aprimorar as dinâmicas de direcionamento das marcas. O Google
defende que tem política clara contra conteúdo enganoso e que suas
plataformas são orientadas a trabalhar com fontes confiáveis.
%ñ encontrei: (PIRES, 2020).

O Sleeping Giants também pode funcionar didaticamente para que as
empresas repensem seus modelos de negócios, considerando que é preciso
ampliar as escolhas dos espaços publicitários para além das
visualizações dos anúncios e que é preciso identificar os conteúdos dos
sites para ver se estão em consonância com os valores da empresa e com
os preceitos democráticos.

O Sleeping Giants Brasil foi criado em 2016 mas passou um bom tempo
desativado até que, em 2020, um casal de jovens, estudantes de Direito,
resolveram ativar a iniciativa. Na primeira postagem que fizeram no
Twitter com o alerta a uma empresa obtiveram em apenas uma semana 300
mil seguidores, mostrando que no país há espaço para a iniciativas que
tenham como objetivo enfrentar a desinformação e o discurso de ódio.

\subsection{Stop hate for profit}

Há tempos que o Facebook é questionado por diversos setores da sociedade
civil sobre suas práticas na mais importante rede social do planeta.
Embora tenha Termos de Uso para os usuários fazerem parte da plataforma,
na maioria das vezes, a aplicação das medidas não é clara e
transparente.

Após os escândalos da Cambridge Analytica, conhecemos um pouco melhor
essa face da plataforma, que forneceu dados de aproximadamente 50
milhões de usuários para campanhas políticas, sem que houvesse qualquer
tipo de consentimento para tal uso.

Os dados coletados pelo Facebook foram utilizados por um aplicativo de
teste psicológico e os usuários que participaram entregaram
seus dados e os de seus amigos que também utilizam a rede social. O
aplicativo que realizou a coleta das
informações (\textit{thisisyourdigitallife}) foi desenvolvido por
Aleksandr Kogan, pesquisador da Universidade de Cambridge, do Reino
Unido. Os dados coletados revelam aspectos da identidade das pessoas
como nome, profissão, local de residência, gostos e hábitos, além da
rede de contatos pessoais. Esse conjunto de informações foi utilizado
para eleger o presidente Donald Trump e para influenciar os eleitores
britânicos a escolherem a saída da Inglaterra do bloco europeu.

A denúncia realizada em 2018, além de cair como uma bomba nos meios
políticos pelo uso inescrupuloso dos dados pessoais, também demonstrava
que a rede social não se preocupava com a proteção dos dados de seus
usuários, conforme consta em seus Termos de Uso. A situação para o
Facebook ficou ainda mais delicada na época, considerando que já tinha
sido questionado sobre o uso da plataforma para a proliferação de
notícias fraudulentas durante nas eleições norte-americanas de 2016.

O impacto foi imediato sobre a imagem da plataforma que à época perdeu
aproximadamente 11,5 bilhões de reais na bolsa de valores norte-americana, ou 
35 bilhões de dólares. Apesar da perda, sabe-se que a empresa mantém altas taxas de
lucro e um dos maiores patrimônios financeiros do mundo.

Mark Zuckerberg, um dos fundadores e \textsc{ceo} da rede social, sempre se
esquivou das denúncias, tendo prestado depoimento no Congresso do
Estados Unidos para esclarecer o papel da empresa no vazamento de dados
de usuários. Zuckerberg pediu desculpas por não ter conseguido
desenvolver mecanismos para impedir a propagação de notícias
fraudulentas que tenham influenciado em processos políticos. Cabe
ressaltar que, apesar das desculpas, a plataforma não tinha atitude
realmente efetiva para enfrentar a propagação de conteúdos falsos.

Nesse contexto, surge a campanha \#Stophateforprofit.\footnote{Em português, \textit{Pare de lucrar com o ódio}.} Trata-se de uma iniciativa que
criou uma coalizão de organizações de direitos civis tais como \textsc{adl} (\textit{The Anti-Defamation League}), \textsc{naacp}, Sleeping Giants, Color Of
Change, Free Press e Common Sense que pediam para que as empresas
parassem de anunciar no Facebook durante o mês de julho de 2020, como
forma de pressionar a rede social a tomar medidas efetivas para impedir
a programação de discursos de ódio, desinformação, racismo e outras
formas de intolerância.

Para a coalização, o Facebook arrecada bilhões com seu império de
publicidade e a campanha tinha o objetivo de estimular as empresas a
retirarem o dinheiro investido em anúncios na plataforma que, segundo os
articuladores, vem sendo omissa em relação ao enfrentamento da
desinformação.

Existem outras iniciativas em diversos países, articulando grupos e
segmentos da sociedade civil com o objetivo de impedir a continuidade e
ampliação da disseminação de notícias falsas, desinformação e discurso
de ódio nas plataformas digitais. Seguramente, ainda teremos um longo
caminho para reverter essa realidade.

\chapter{A outra face da pandemia}

Ao começarmos os primeiros esboços dessa reflexão tínhamos um plano
original, substancialmente alterado em virtude da pandemia que trouxe ao
centro do debate alguns acontecimentos relacionados às preocupações de
diversos setores da sociedade em relação ao crescimento das notícias
falsas e da desinformação.

As primeiras informações sobre a pandemia começaram a chegar em
fevereiro no país, mas eram muito difusas e então realizamos o carnaval,
uma das mais importantes festas populares,~com aglomeração e sem a
adoção de medidas sanitárias.~Rapidamente essa realidade se transformou
e começávamos a receber notícias de mortes de milhares de pessoas em
outros países e por aqui apareciam os primeiros casos.

Epidemiologistas, virologistas e outras autoridades sanitárias corriam
para compreender o comportamento do novo coronavírus e poder atuar para
conter o seu avanço. A comunidade científica internacional buscava
soluções para o tratamento dos infectados, mas tudo era muito novo e não
apareciam soluções rápidas, mantendo um ambiente de incertezas em
relação às formas de contágio, prevenção e tratamentos.

Nesse cenário, começava a surgir a outra face da pandemia: a
desinformação, as teorias da conspiração, o negacionismo, as notícias
falsas e as informações fraudulentas que tomavam conta das redes sociais
e se espalhavam com grande velocidade, confundindo e assustando ainda
mais a população.

O crescimento do compartilhamento de conteúdos sem nenhum tipo de
evidência científica impactava significativamente o debate público sobre
as medidas necessárias para impedir o avanço e os impactos do
coronavírus. Confundir e deslegitimar as informações das autoridades
sanitárias reconhecidas internacionalmente era um dos objetivos centrais
por parte dos setores que comungam do mesmo negacionismo científico dos
integrantes do governo de Jair Bolsonaro. Acreditamos que faz parte de
uma estratégia mais ampla, ou seja, não são dinâmicas separadas e em
diversas situações é possível identificar claramente que o ato de
colocar em suspeição uma vacina é frequentemente associado a outros
valores, tais como a xenofobia, a crítica aos modos de vida dos
chineses, a crítica ao comunismo da China, portanto, podemos afirmar que
se trata de comportamento frequentemente adotado por grupos e lideranças
políticas com caráter notadamente antidemocráticos.

O crescente uso das plataformas digitais para as práticas que
transformaram indivíduos e grupos em informações potencialmente
monetizáveis é motivo de preocupação, considerando que essas empresas
criaram um mercado para compra e venda de bases de dados que podem ser
utilizados tanto para objetivos financeiros quanto políticos, conforme
abordamos no capítulo \textsc{ii}, ao analisarmos os escândalos envolvendo a
empresa Cambridge Analytica que usou ilegalmente dados de eleitores
norte-americanos para influenciar o voto durante as eleições presidenciais de
2016, favorecendo a candidatura vitoriosa de Donald Trump, do Partido
Republicano.

As revelações sobre as práticas da Cambridge Analytica comprovam que a
economia informacional está cada vez mais amparada no chamado mercado de
dados. Obtidos pelas chamadas pegadas digitais, nossos dados são
capturados pelas tecnologias cibernéticas que também têm ampla
capacidade de armazenamento e análise que podem ter diversos tipos de
utilização.

Essa apropriação desloca o debate da informação da cooperação, da
colaboração para o compartilhamento com finalidade de extração de lucros
para empresas. Não é à toa que o capitalismo de plataforma seja liderado
pelas gigantes do setor, as \textsc{gafa}s: Google, Apple, Facebook e Amazon.

A ampliação da visibilidade das teorias da conspiração, possibilitada
pelas redes digitais, chegaram a patamares quase impensáveis e
expressavam não haver limites para a construção e a potencialização de
mitos e medos existentes na sociedade, e funcionavam como verdadeiras
câmeras de eco que faziam com que as bolhas se autocentrassem ainda
mais, construindo um contexto de polarização entre os que
``acreditavam'' no vírus e os que achavam que se tratava de uma
conspiração para afetar a população brasileira e, principalmente, o
governo federal.

A ideologização em torno da pandemia levava a confrontos constantes
entre governadores e prefeitos que cobravam medidas por parte do governo
federal. Disputas político-partidárias com vistas aos processos
eleitorais futuros agitavam o debate para ver qual narrativa seria
\textit{vitoriosa} sobre a pandemia. As redes reverberavam essas disputas com
uma enxurrada de desinformação, \textit{memes} e notícias falsas que proliferavam
nas redes.

O discurso do presidente era uma espécie de senha para movimentar a rede
de desinformação. Quando falava contra o isolamento social para evitar
aglomerações, seus grupos de apoiadores organizavam manifestações de rua
contra as medidas de isolamento social decretadas por prefeitos e
governadores, potencializando não somente a ampliação do contágio, mas
também um ambiente de conflitos num momento em que era necessária a união
a fim de mais rapidamente se conter o avanço do contágio.

Ignorando o papel que um líder político ocupa, Bolsonaro, além de
promover aglomerações, frequentemente aparecia em público sem máscara,
chegando a divulgar um suposto estudo de alguma universidade internacional
que apresentava os malefícios do equipamento de proteção. Para ele, a doença
não passava de uma ``gripezinha'', minimizando os riscos para o aumento de
infectados e, consequentemente, o colapso do sistema de saúde.

O tratamento precoce, composto por cloroquina e outros medicamentos,
também era propagandeado pelo presidente e os técnicos que atuavam como
consultores informais, uma espécie de estrutura paralela ao Ministério
da Saúde que influenciava nas decisões governamentais e no discurso
adotado pelo governante que não desistia dessa recomendação, mesmo após
a comprovação científica de sua ineficácia.

Começava a ficar claro que a desinformação era parte da estratégia da
ação política governamental e contava com a criação de estruturas
paralelas que atuavam dentro das instituições usando a estrutura e os
recursos públicos para divulgar mentiras e informações fraudulentas.
Havia uma inversão, considerando que as estruturas oficiais passavam a
não ser a referência para as definições das políticas governamentais.

Além da atuação na área da saúde, as estruturas paralelas atuavam também
em outros órgãos públicos federais, como no Ministério do Meio Ambiente,
que realizou um verdadeiro desmonte nos órgãos de proteção ambiental e
bloqueou a participação da sociedade civil, facilitando medidas de
exportação de madeira ilegal, atividades de garimpo em terras indígenas,
entre outros ataques ao meio ambiente. Conforme o próprio ministro
afirmou, era a hora de aproveitar que todos estavam preocupados com a
pandemia para aprovar a flexibilização das regras de proteção ao meio
ambiente.

O discurso de ódio, o anti-intelectualismo e o anticientificismo eram
calculadamente organizados nas práticas discursivas de Bolsonaro, e a
retórica antivacina esteve presente não somente nas falas presidenciais,
mas também na negligência e morosidade na aquisição dos imunizantes. Aos
poucos, era possível perceber que não se tratava somente de
negacionismo, mas da comercialização de imunizantes superfaturados por
parte de bolsonaristas influentes no governo. Ficava difícil sustentar o
discurso contra corrupção que garantiu a vitória em 2018, considerando
que desde o início do governo não paravam de aparecer denúncias contra o
presidente, seus filhos e apoiadores mais fiéis. Começava a ficar
insustentável a base de apoiadores somente com a desinformação, o
discurso de ódio e as teorias da conspiração, embora o presidente não
fizesse nenhum sinal que mudaria o rumo de sua narrativa.

Outro mecanismo era a criação de termos, ou seja, neologismos que ao
serem pesquisados nos sites de buscas direcionavam os leitores para
portais identificados como produtores e disseminadores de notícias
falsas. A sofisticação do processo desinformativo ficava mais clara dia
após dia.

Como dizia Hannah Arendt, a verdade e a política nunca tiveram boas
relações. Isso já é sabido, mas não paira dúvida que desde o início do
governo Bolsonaro entrávamos em num novo patamar dessa relação tão
delicada.

Concluímos essa reflexão ainda com muitas incertezas em relação ao
futuro e aos desdobramentos das investigações sobre as denúncias
mencionadas, mas já é possível ter clareza que a desinformação era parte
constitutiva dos procedimentos governamentais, produzindo inestimável
impacto na sociedade democrática.

\chapter{Referências bibliográficas}

\begin{bibliohedra}
\tit{adorno}, Theodor W. \textit{Estudos sobre a personalidade autoritária}. São Paulo: Editora Unesp, 2019.

\titidem. Educação após Auschwitz. \textit{In}: \line(1,0){25}. \textit{Educação e emancipação}. São Paulo: Paz \& Terra, 2020.

\tit{allcott}, H. e \textsc{gentzkow}, M. ``Social media and fake news in the
2016 election''. \textit{In: Journal of Economic Perspectives}, v.\,31, n.\,2, p.\,211--236, 2017.

\tit{almeida}, Silvana L. \textit{et al.} WhatsApp: a desordem da informação na
eleição presidencial brasileira de 2018, \textit{Anais \textsc{vii} Simpósio
Internacional Lavits}, Salvador, junho de 2019.

\tit{arendt}, Hannah. \textit{Entre o passado e o futuro}. São Paulo: Editora
Perspectiva, 1997.

\tit{asano}, Camila Lissa \textit{et al.} Boletim n.\,10: Direitos na Pandemia --
mapeamento e análise das normas jurídicas de resposta à \textsc{covid-19} no
Brasil. \textit{\textsc{cepedisa}} e \textit{Conectas}, São Paulo, 20 de janeiro de 2021.

\tit{bezerra}, Arthur, \textsc{capurro}, Rafael e \textsc{schneider}, Marco. Regimes de
verdade e poder: dos tempos modernos à era digital. \textit{Liincem Revista}, Rio de Janeiro, v.\,13, n.\,2, p.\,371--380, novembro de 2017.

\tit{bounegru}, Liliana \textit{et al.}
\textit{A Field Guide to Fake News: a collection of recipes for those
who love to cook with digital methods}. Amsterdã: Public Data Lab, 2017.

\tit{bramati}, Daniel, \textsc{monnerat}, Alessandra e \textsc{brenbati}, Katia.
Cloroquina tem Bolsonaro como maior influenciador do mundo. \textit{O Estado de S.\,Paulo}, 6 de junho de 2021.

\tit{cgi.br}. Comitê Gestor da Internet no Brasil. \textit{Relatório Internet,
Desinformação e Democracia}.

\tit{chicarino}, Tathiana S. \textit{et. al.} Como dois estudantes de 22 anos puseram de pé o Sleeping Giants Brasil. \textit{Aurora: revista de arte, mídia e política}, São Paulo, v.\,14, n.\,40, p.\,6--27, fevereiro--maio de 2021.

\tit{danowski}, Deborah. \textit{Negacionismos}. São Paulo: n-1 edições, 2018.

\tit{darnton}, Robert. A verdadeira história das notícias falsas. \textit{El
País}, 30 de abril de 2017.

\tit{datafolha}. \textit{Relatório Uso das redes sociais}. Instituto
Datafolha: Eleições 2018.

\tit{Datafolha}: 73\% dizem que pretendem tomar vacina contra
\textsc{covid-19}, e 56\% acham que ela deve ser obrigatória. \textit{G1}, 12 de dezembro de 2020.

\tit{davis}, Mike. O coronavírus e a luta de classes: o monstro bate
à nossa porta. \textit{Contee}, 18 de março de 2020.

\tit{dieese}. Departamento intersindical de estatística e estudos socioeconômicos. \textit{Custo da cesta básica aumenta em 10 capitais}. São Paulo, 5 de março 2020.

\tit{elias}, Norbert e \textsc{scotson}, Jhon. \textit{Estabelecidos e outsiders}.
Trad. Vera Ribeiro e Pedro Süssekind. Rio de Janeiro: Zahar, 1994.

\tit{ferraz}, Adriana. Bolsonaro diz que é preciso ``enfrentar vírus como homem e não como moleque''. \textit{Uol}, 29 de março de 2020.

\tit{foucault}, Michel. \textit{O nascimento da biopolítica}. São Paulo:
Martins Fontes, 2008.

\tit{frisch}, Felipe. 86\% conhecem alguém que morreu de covid no Brasil. \textit{Valor Econômico}, 27 de abril de 2021.

\tit{gaglioni}, Cesar. O que há sobre o Brasil nos documentos da
Cambridge Analytica. \textit{Nexo}, 6 de janeiro de 2020.

\tit{global disinformation index}. \textit{The Quarter Billion Dollar
Question: How is Disinformation Gaming Ad Tech?}, Reino Unido, setembro de 2019.

\tit{grandin} Felipe. \textsc{é \#fake} que pesquisa recente indique a
hidroxicloroquina como o tratamento mais eficaz contra o coronavírus. \textit{G1}, 21 de maio de 2020.

\tit{heclo}, H. Campaigning and governing: a conspectus. \textit{In}: \textsc{ornstein}, N.\,J. e \textsc{mann}, T.\,E. \textit{The Permanent Campaign and Its Future}. Washington:
American Enterprise Institute and The Brookings Institution, 2000.

\tit{libório}, Barbara e \textsc{cunha}, Ana R. Notícias falsas foram
compartilhadas ao menos 3,84 milhões de vezes durante as eleições. \textit{Aos
Fatos}, 31 de outubro de 2018.

\tit{lupa}. Verifica Coronavírus \textsc{ep05}: a onda de desinformação nos
registros de mortes por \textsc{covid-19}. \textit{Piaui}, Rio de Janeiro, 14 de maio de 2020.

\tit{machado}, Caio e \textsc{konopacki}, Marco. \textit{Poder computacional: automação
no uso do whatsapp nas eleições: estudo sobre o uso de ferramentas de
automação para o impulsionamento digital de campanhas políticas nas
eleições brasileiras de 2018}. Rio de Janeiro: Instituto de Tecnologia e Sociedade do Rio de Janeiro, 2019.

\tit{mazzo}, Aline. 91\% dos brasileiros pretendem se vacinar ou já se vacinaram, mostra Datafolha. \textit{Folha de S.\,Paulo}, 18 de maio de 2021.

\tit{mbembe} Achille. \textit{Necropolítica}. Trad. Renata Santine. São Paulo: n-1 edições, 2018.

\titidem. Pandemia democratizou poder de matar, diz autor
da teoria da \textit{necropolítica}. \textit{Folha de S.\,Paulo}, 30 de março de 2020.

\tit{melo}, Patrícia C. Empresários bancam campanha contra o \textsc{pt} pelo
WhatsAppp. \textit{Folha de S.\,Paulo}, 18 de outubro de 2018.

\tit{menezes}, Luiz Fernando. Nobel de Medicina não disse que novo
coronavírus foi criado pela China. \textit{Aos Fatos}, 28 de abril de 2020.

\tit{ministerio da saúde}. \textit{O que você precisa saber sobre o
coronavirus}. Disponível em: \textit{coronavirus.saude.gov.br}.
%Acesso em: 14 jun. 2020.

\tit{moraes} Mauricio. Brasil lidera desinformação sobre número de
casos e mortes por \textsc{covid-19} no mundo. \textit{Uol}, 10 de junho de 2020.

\tit{morozov}, Evgeny. \textit{Big Tech: a ascensão dos dados e a
morte da política}. São Paulo: Ubu, 2018.

\tit{nielsen}, Jakob. Website Reading: It (Sometimes) Does Happen. \textit{Nielsen Norman Group}, 24 de junho de 2013. 

\tit{opas}. Organização Pan-Americana da Saúde. \textit{Entenda a infodemina e a desinformação na luta contra o \textsc{covid-19}}. Relatório do Departamento de Evidência para ação em saúde, 2020.

\tit{parisier}, Eli. \textit{O filtro invisível: o que a internet está
escondendo de você}. Rio de Janeiro: Zahar, 2012.

\tit{piero}, Bruno de.
Epidemia de fake news: Organização Mundial da Saúde chama a
atenção para grande circulação de notícias falsas sobre o novo
coronavírus. \textit{Revista Pesquisa \textsc{fapesp}}, 7 de abril de 2020.

\tit{posetti}, Julie e \textsc{matthews}, Alice. Una breve guía de la historia
de las `noticias falsas' y la desinformación. \textit{\textsc{icfj}}, julho de 2018.

\tit{projeto comprova}. Ao contrário do que afirma blog, \textsc{oms} recomenda isolamento como uma das medidas de combate ao novo coronavírus. 6 de maio de 2020.

\titidem Vacina do coronavírus não terá microchip para rastrear a população. 6 de abril de 2020. 

\tit{rudnitzki}, Ethel e \textsc{scofield}, Laura. Robôs levantaram hashtag que
acusa China pelo coronavírus. \textit{A Pública}, 20 de março de 2020.

\tit{runciman}, David. \textit{Como a democracia chega ao fim}. São Paulo:
Todavia, 2018.

\tit{scamparini}, Ilze. ``Todo fundamentalismo quase sempre se baseia em afirmações falsas''. \textit{Consultório Jurídico}, 17 de julho de 2015.

\tit{sena jr.}, Carlos Zacarias de. Obscurantismo e anticientificismo no
Brasil bolsonarista: anotações sobre a investida protofascista contra a
inteligência e a ciência no Brasil. \textit{Cadernos \textsc{gposshe} On-line},
Fortaleza, v.\,2, n.\,Especial, 2019.

\tit{silveira}, Sergio Amadeu da. \textit{Democracia e os códigos
invisíveis: como os algoritmos estão modulando comportamentos e escolhas
política}. São Paulo: Edições Sesc, 2019.

\tit{silverman}, Craig. This Analysis Shows How Fake Election News
Stories Outperformed Real News On Facebook. \textit{Buzzfeed News}, 16 de novembro de 2016.

\tit{srnicek}, Nick. \textit{Platform Capitalism}. Cambridge: Polity Press, 2016.

\tit{tribunal superior eleitoral}. \textit{Processo: 0601530-54.2018.6.00.0000}. Representação, 2018.

\tit{ujvari}, Stefan Cunha. \textit{A história da humanidade contada pelos vírus}. São Paulo:
Editora Contexto, 2012.

\tit{wardle}, Claire. Fake news. It’s complicated. \textit{First Draft}, 16 de fevereiro de 2017.

\titidem\mbox{} e \textsc{derakhshan}, Hossein. \textit{Information
Disorder: Toward an interdisciplinary framework for research and policy
making}. Estrasburgo: Council of Europe, 2017.
\end{bibliohedra}
